\documentclass[A4, 12pt, twoside]{article}
\usepackage{amsmath,amsfonts,amsthm,amssymb,amscd}
 \usepackage{amssymb}
 \usepackage[cp1251]{inputenc}
 \usepackage[T2A]{fontenc}
 \usepackage[all]{xy}
 \usepackage{latexsym}
 \usepackage{srcltx}
 \usepackage{amsmath}
 \usepackage{amsfonts}
 \usepackage{amscd}
 \usepackage{amssymb}
 \usepackage{amsthm}
 \usepackage{euscript}
 \usepackage{url}
 \usepackage{amssymb, amsthm}
 \usepackage{mathrsfs}
 \usepackage{srcltx}
 \usepackage{hyperref}\usepackage{setspace}
 \usepackage[russian,english]{babel}
% \usepackage[pdftex]{graphicx}%
 \usepackage{mathrsfs}
% \usepackage{graphicx}


 \newcommand{\refen}{\vspace{10mm} \noindent{\Large{\bf References}}}


 \sloppy
 \paperwidth=210mm
 \paperheight=297mm
 \textheight=250mm
 \textwidth=165mm
 \oddsidemargin=-0.4mm
 \evensidemargin=-5.4mm
 \topmargin=-5.4mm
 \footskip=12.5mm
 \headsep=0mm
 \headheight=0mm


 \begin{document}


\begin{center} \textbf{
	Termination of original F5 \\[.3cm]
	V.V.~Galkin}  (MSU, Moscow) \\
	\emph{E-mail address}: \texttt{galkin-vv@yandex.ru}\\[.3cm]
\end{center}
\normalsize

The original F5 algorithm for computing Gr\:{o}bner bases was introduced by Faug\`{e}re in 2002.
It is formulated for ideals generated by a finite sequence of homogeneous polynomials.
The result correctness is shown for any input that terminates the algorithm,
but the termination itself was proved only for the case of input being regular polynomial sequence.

The proof of algorithm termination for any homogeneous input without reference to regularity is presented.
It is shown that if the algorithm doesn't terminate it generates polynomials $p_1, p_2$, where $p_2$ signature-safe reduces $p_1$,
and for such pair algorithm already generated $p_3$ that signature-safe reduces $p_1$ and passes algorithm criteria, which is a contradiction.

%%%%%%%%%%%%%%%%%%%%%%%%%%%%%%%%%%%%%%%%%%%%%%%%%%%%%%%%%%%%%%%

%%%%%%   Далее разместите аннотацию на русском языке   %%%%%%%%

%%%%%%%%%%%%%%%%%%%%%%%%%%%%%%%%%%%%%%%%%%%%%%%%%%%%%%%%%%%%%%%

\begin{center} \textbf{
	Остановка алгоритма F5 \\[.3cm]
	В.В.~Галкин}  (МГУ, Москва) \\
	\emph{E-mail address}: \texttt{galkin-vv@yandex.ru}\\[.3cm]
\end{center}
\normalsize
Исходный алгоритм F5 вычисления базисов Грёбнера бфл предложен Фожером в 2002 г.
Он сформулирован для иеалов, породжённых конечной последовательностью однородных многочленов.
Корректность результата доказана для всех случаев, когда алгоритм останавливается,
однако сама остановка была доказана лишь для случая, когда на входная последовательность многочленов регулярна.

Предлагается доказательство остановки алгоритма для любой последовательности однородных многочленов, без какого-либо использования регулярности.
Схема доказательства состоит из двух частей: в первой части показано, что если алгоритм
не останавливается, то рано или поздно он получит пару многочленов,
в которой первый может редуцировать второй. При этом, однако, не утверждается,
что такая редукция будет разрешена критериями, предложенными в F5.
Вторая часть показывает, что при существовании такой пары также будет
существовать пара в которой редукция разрешена всеми критериями. Существование
такой пары приводит к противоречию.

\end{document}
