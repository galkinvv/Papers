\documentclass[A4, 12pt, twoside]{article}
\usepackage{amsmath,amsfonts,amsthm,amssymb,amscd}
 \usepackage{amssymb}
 \usepackage[cp1251]{inputenc}
 \usepackage[T2A]{fontenc}
 \usepackage[all]{xy}
 \usepackage{latexsym}
 \usepackage{srcltx}
 \usepackage{amsmath}
 \usepackage{amsfonts}
 \usepackage{amscd}
 \usepackage{amssymb}
 \usepackage{amsthm}
 \usepackage{euscript}
 \usepackage{url}
 \usepackage{amssymb, amsthm}
 \usepackage{mathrsfs}
 \usepackage{srcltx}
 \usepackage{hyperref}\usepackage{setspace}
 \usepackage[russian,english]{babel}
% \usepackage[pdftex]{graphicx}%
 \usepackage{mathrsfs}
% \usepackage{graphicx}


 \newcommand{\refen}{\vspace{10mm} \noindent{\Large{\bf References}}}


 \sloppy
 \paperwidth=210mm
 \paperheight=297mm
 \textheight=250mm
 \textwidth=165mm
 \oddsidemargin=-0.4mm
 \evensidemargin=-5.4mm
 \topmargin=-5.4mm
 \footskip=12.5mm
 \headsep=0mm
 \headheight=0mm


 \begin{document}


\begin{center} \textbf{
  Termination of original F5 \\[.3cm]
	V.V.~Galkin}  (MSU, Moscow) \\
	\emph{E-mail address}: \texttt{galkin-vv@yandex.ru}\\[.3cm]
\end{center}
\normalsize

The original F5 algorithm introduced by Faug\`ere is formulated for
any homogeneous polynomial set input. The correctness of output is
shown for any input that terminates the algorithm, but the termination
itself is proved only for the case of input being regular polynomial
sequence. This article shows that algorithm correctly terminates for
any homogeneous input without any reference to regularity. The scheme
contains two steps: first it is shown that if the algorithm does not
terminate it eventually generates two polynomials where first is a
reductor for the second. But first step does not show that this reduction
is permitted by criteria introduced in F5. The second step shows that
if such pair exists then there exists another pair for which the reduction
is permitted by all criteria. Existence of such pair leads to contradiction.

%%%%%%%%%%%%%%%%%%%%%%%%%%%%%%%%%%%%%%%%%%%%%%%%%%%%%%%%%%%%%%%

%%%%%%   Далее разместите аннотацию на русском языке   %%%%%%%%

%%%%%%%%%%%%%%%%%%%%%%%%%%%%%%%%%%%%%%%%%%%%%%%%%%%%%%%%%%%%%%%

\begin{center} \textbf{
	Остановка алгоритма F5 \\[.3cm]
	В.В.~Галкин}  (МГУ, Москва) \\
	\emph{E-mail address}: \texttt{galkin-vv@yandex.ru}\\[.3cm]
\end{center}
\normalsize
Алгоритм F5, предложенный Фожером, принимает в качестве входных данных
произвольное множество однородных многочленов и корректность результата
доказана для всех случаев, когда алгоритм останавливается. Однако
остановка алгоритма за конечное число шагов доказана лишь для случая,
когда на вход алгоритма подаётся регулярная последовательность многочленов.
В этой работе показано, что алгоритм останавливается на любых входных
данных без какого-либо использования регулярности. Схема доказательства
состоит из двух частей: в первой части показано, что если алгоритм
не останавливается, то рано или поздно он получит пару многочленов,
в которой первый может редуцировать второй. При этом, однако, не утверждается,
что такая редукция будет разрешена критериями, предложенными в F5.
Вторая часть показывает, что при существовании такой пары также будет
существовать пара в которой редукция разрешена всеми критериями. Существование
такой пары приводит к противоречию.

\end{document}
