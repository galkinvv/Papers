%% LyX 2.0.5.1 created this file.  For more info, see http://www.lyx.org/.
%% Do not edit unless you really know what you are doing.
\documentclass[english]{birkmult}
\usepackage[T1]{fontenc}
\usepackage[utf8]{inputenc}
\usepackage{babel}
\usepackage{amsthm}
\usepackage{amsmath}
\usepackage{amssymb}
\usepackage[unicode=true,pdfusetitle,
 bookmarks=true,bookmarksnumbered=true,bookmarksopen=true,bookmarksopenlevel=7,
 breaklinks=false,pdfborder={0 0 0},backref=false,colorlinks=false]
 {hyperref}
\hypersetup{
 unicode=false}

\makeatletter
%%%%%%%%%%%%%%%%%%%%%%%%%%%%%% Textclass specific LaTeX commands.
\newenvironment{lyxlist}[1]
{\begin{list}{}
{\settowidth{\labelwidth}{#1}
 \setlength{\leftmargin}{\labelwidth}
 \addtolength{\leftmargin}{\labelsep}
 \renewcommand{\makelabel}[1]{##1\hfil}}}
{\end{list}}

\@ifundefined{date}{}{\date{}}
%%%%%%%%%%%%%%%%%%%%%%%%%%%%%% User specified LaTeX commands.
\sloppy
%\usepackage{geometry}

\makeatother

\begin{document}

\author{Vasily Galkin}
\address{Chair of Higher Algebra, Faculty of Mechanics and Mathematics, Moscow State University, Moscow, Russia}
\email{galkin-vv@ya.ru}


\title{Signature-based Gröbner basis computations for approximate input}
\maketitle
\begin{abstract}
This work develops methods for applying signature-based Gröbner basis
computation algorithms to polynomial systems with inexact input data
in $\mathbb{C}$. First method is based on the idea of modular computations
which is known to return incorrect result for small enough modules.
This is solved by giving strictly estimation of such possibility.
Second method is based on the previously known TSV (Term Substitution
with Variables) strategy. It proposes the signature ordering that
allows TSV to be applied to signature-based algorithms without restarts.
\end{abstract}
\global\long\def\HM{\mathrm{LM}}


\global\long\def\HC{\mathrm{LC}}


\global\long\def\Sss#1#2{\left\langle #1\centerdot#2\right\rangle }



\section{Algorithms for computing Gröbner basis}

One of the modern practical applications of commutative algebra is
its usage in computer solving of approximate polynomial systems arising
from various areas. The Gröbner basis is one of the main instruments
that can be used to solve them. The classic algorithm for computing
Gröbner bases is the Buchberger algorithm, but there are more efficient
algorithms including signature-based algorithms \cite{FaugereF5,GVW,GroebnerInExtFields}.

Most of them are formulated for coefficients in a field, but not for
approximate numbers. The root of problem with applying to approximate
numbers is in zero equality testings: they are needed by algorithms
for key branchings but not always defined for approximate numbers.
There are different approaches for this problem:
\begin{enumerate}
\item \label{enu:in-Q}round inexact numbers to rational and perform the
computation in field $\mathbb{Q}$;
\item \label{enu:as-symbols}treat approximate numbers as symbols and use
symbolic quotient field;
\item \label{enu:other- basis}abandon the Gröbner basis and use other variant
of basis for system solving;
\item use a pseudo-field of approximate numbers but avoid problematic zero
testing:

\begin{enumerate}
\item \label{enu:other-fields}get zero testing result with some heuristics
\cite{ModularTraverso}.
\item \label{enu:adjusting-the-monomial}add special polynomials to reduce
problematic coefficient \cite{GroebnerInExtFields}.
\end{enumerate}
\end{enumerate}
\pagebreak{}

Approaches \ref{enu:in-Q} and \ref{enu:as-symbols} make the approximate
variant much slower than the same algorithm applied to finite fields
because non-constant complexity of arithmetic operations with coefficients.
The possibility to combine approach \ref{enu:other- basis} with a
signature-based algorithms is an open question because they are initially
designed for Gröbner basis.

The approach \ref{enu:other-fields} as it is described in \cite{ModularTraverso}
is perfect in terms of computational speed. But the heuristics can
lead to wrong answer and there is no any probability estimations of
such case, so while the approach may be used in practice, it lacks
the strict proof. In \cite{GroebnerInExtFields} the TSV method based
on approach \ref{enu:adjusting-the-monomial} was successfully combined
with signature-based algorithm F5. But it has a serious slowdown compared
to original F5: after every polynomial addition the F5 is restarted
from the beginning because signature based algorithms require particular
order in input polynomial processing. This work introduces methods
based on approaches \ref{enu:other-fields}, \ref{enu:adjusting-the-monomial}
that try to solve above mentioned problems.


\section{Problem statement in approximate numbers}

Before formulating these methods we need to prepare the formal statement
of the problem. We assume that approximate input data consists of
a multivariate polynomial equations system with complex coefficients
and for each coefficient we know the maximum inaccuracy of it's value.
Approximate numbers are defined in a way that such input data can
be easily represented by them. The \emph{approximate number} is a
pair $\left(a,\varepsilon\right)$, where $a\in\mathbb{C},0\leqslant\varepsilon\in\mathbb{R}$.
Element $a$ is called \emph{approximation}, and $\varepsilon$ --
\emph{variation radius}. The elements of set $\left\{ a+e|e\in\mathbb{C},\left|e\right|\leqslant\varepsilon\right\} \subset\mathbb{C}$
are called \emph{specializations} of approximate number.

For every approximate number in input we can take some specialization
and get the polynomial system in $\mathbb{C}$. Such systems are called
\emph{specializations of approximate input data}. The desired approximate
solution of the problem consists of the set of approximate numbers
corresponding to variables. \emph{Specializations of solution} are
defined too -- they are represented by a set of variable values in
$\mathbb{C}$. The design of approximate computations is based on
the following \emph{approximation rule}: the approximate result of
some computation should have specializations equal to any of exact
results corresponding to computation in $\mathbb{C}$ performed with
some input data specialization. Applying this to the whole task we
get: $\forall$ specializations of approximate input data $\exists$
specialization of approximate solution that gives the exact solution
in $\mathbb{C}$ for the exact input data specialization. Applying
the rule to arithmetic operations we get their definitions:
\begin{lyxlist}{Multiplication}
\item [{Addition}] $(a_{1},\varepsilon_{1})+(a_{2},\varepsilon_{2}):=(a_{1}+a_{2},\varepsilon_{1}+\varepsilon_{2})$.
\item [{Multiplication}] $(a_{1},\varepsilon_{1})\times(a_{2},\varepsilon_{2}):=(a_{1}a_{2},\varepsilon_{1}\left|a_{2}\right|+\varepsilon_{2}\left|a_{1}\right|+\varepsilon_{1}\varepsilon_{2})$.
\item [{Negation}] based on multiplication: $-(a,\varepsilon):=(a,\varepsilon)\times\left(-1,0\right)=(-a,\varepsilon).$
\item [{Inversion}] defined only if zero isn't $(a,\varepsilon)$ specialization
($\Longleftrightarrow\left|a\right|-\varepsilon>0$):
\end{lyxlist}
\[
\frac{1}{(a,\varepsilon)}:=\left(\frac{1}{a},\frac{\varepsilon}{\left|a\right|\left(\left|a\right|-\varepsilon\right)}\right).
\]





\section{Non-invertible approximate elements}

 The comparison of same algorithm runs on the approximate input and
it's specializations reveals significant difference in two cases:
\begin{itemize}
\item approximated algorithm attempts to perform reduction by polynomial
with non-invertible leading coefficient;
\item leading coefficient for approximate case is zero for the specialization.
It is non-invertible because has a zero specialization due to approximation
rule.
\end{itemize}
In both cases the difference is inspired by non-invertible approximate
elements. They can be classified to several kinds:
\begin{itemize}
\item element is \emph{symbolic zero} if the corresponding $\mathbb{C}$-coefficient
is zero in the algorithm application to any of input specializations;
\item element is \emph{computation-introduced zero} if the corresponding
$\mathbb{C}$-coefficient is non-zero in the algorithm application
to any of input specializations;
\item in the other case the element is called \emph{input-inspired zero.}
\end{itemize}

\subsection{Modular method}

The first two kinds of non-invertible elements correspond to a situation
in which there is a chance to fix non-invertible approximate element
by changing it to zero (for symbolic zero) or to some invertible element
(for\emph{ }computation-introduced zero) and still keep the approximation
rule satisfied. The trivial approach to determine possibility of such
replacement is keeping a representation of each coefficient as symbolic
expression of input data coefficients and testing if it is identically
equal to zero. The problem of this trivial approach is its slowness.
In \cite{ModularTraverso} an optimization is proposed: replace symbolic
expressions with its values in finite field $\mathbb{F}$. Zeros in
finite field correspond to identically zero expressions with high
probability. Unfortunately the incorrect result probability estimation
is not given.

The proposed modular method is more complex. It allows fixing of some
computation-introduced zeros and gives reasonable estimation for incorrect
result probability, but lacks the time complexity comparison with
other variants.

The symbolic expressions associated with polynomials are used for
calculating how this polynomial is expressed from input data. Their
computed values in $\mathbb{F}$ are not trusted, but are treated
as a suggestion for expressing approximate polynomial with specific
$\HM$ from specific input polynomials. This suggestion is written
as linear system and another finite field $\mathbb{F}'$ is selected.
System solution is performed simultaneously in approximate numbers
and in $\mathbb{F}'$ using the last for detection of zero identical
expressions. The possibility of spurious zeros can be estimated from
system size, known at the moment of $\mathbb{F}'$ selection, so its
prime module can be selected big enough to make error possibility
smaller than fixed ``admissible'' $\alpha$. Another advantage is
in fixing some computation-introduced zeros because fully canceled
input polynomials are not included in the system.


\subsection{Introducing real-weighted order for TSV method}

Modular methods may fail to find some symbolic zeros and can't do
anything with input-inspired zeros. TSV method \cite{GroebnerInExtFields}
solves this problem. Every time the algorithm tries to invert non-invertible
coefficient of monomial $M_{k}$ it adds polynomial $M_{k}-y_{k}$
to input data where $y_{k}$ is a new lesser-than-others monomial.
Such polynomial reduces the $M_{k}$ in other polynomials, so the
their leading monomial changes.

Due to additions the result is a Gröbner basis $G^{E}$ of extended
ideal in an extended polynomial ring $\mathcal{P}^{E}$. Fortunately
the ``action matrix method'' for system solving can be reformulated
to use a normal form operator instead of Gröbner basis. Such operator
can be constructed in $P$ from reduction in $\mathcal{P}^{E}$ by
$G^{E}$.

Other consequence is specific for signature-based algorithms. They
process polynomials strictly by increasing signature but added polynomial
$M_{k}-y_{k}$ should have a signature smaller than current to be
used as reducer. The most widely used signature orders -- POT and
Schreyer -- are quite limited for our case: they don't allow addition
polynomial to have a signature greater than all already processed
polynomials except the current one. So \cite{GroebnerInExtFields}
restarts algorithm after every addition. To allow insertion of signature
just before current we propose a generalized order with parameter
vectors: monomials $w=(w_{1},\dots,w_{m})$ and real numbers $j=(j_{1},\dots,j_{m})$.
Recall that \emph{signature} $\Sss ti$ is a pair of monomial and
integer.
\begin{itemize}
\item \emph{Real-weighted order}:$\Sss{t_{1}}{i_{1}}\prec_{w}\Sss{t_{2}}{i_{2}}\Longleftrightarrow\left[\begin{aligned} & t_{1}w_{i_{1}}\prec t_{2}w_{i_{2}}\\
 & t_{1}w_{i_{1}}=t_{2}w_{i_{2}},j_{i{}_{1}}<j_{i_{2}}
\end{aligned}
\right..$
\end{itemize}
The real-weighted order gives TSV method the ability to add a polynomial
$M_{k}-y_{k}$ to input data with a signature just before the signature
of currently processed polynomial and allows to proceed without algorithm
restart, because it exactly corresponds to the signature that should
be processed now.

\bibliographystyle{plain}
\bibliography{approx_references}

\end{document}
