%% LyX 2.0.2 created this file.  For more info, see http://www.lyx.org/.
%% Do not edit unless you really know what you are doing.
\documentclass[landscape,english,russian]{slides}
\usepackage[T2A]{fontenc}
\usepackage[utf8]{inputenc}
\setcounter{secnumdepth}{1}
\setcounter{tocdepth}{1}
\usepackage{amsmath}
\usepackage{amssymb}

\makeatletter

%%%%%%%%%%%%%%%%%%%%%%%%%%%%%% LyX specific LaTeX commands.
\pdfpageheight\paperheight
\pdfpagewidth\paperwidth

\DeclareRobustCommand{\cyrtext}{%
  \fontencoding{T2A}\selectfont\def\encodingdefault{T2A}}
\DeclareRobustCommand{\textcyr}[1]{\leavevmode{\cyrtext #1}}
\AtBeginDocument{\DeclareFontEncoding{T2A}{}{}}

%% Because html converters don't know tabularnewline
\providecommand{\tabularnewline}{\\}

%%%%%%%%%%%%%%%%%%%%%%%%%%%%%% Textclass specific LaTeX commands.
\newcounter{slidetype}
\setcounter{slidetype}{0}
\newif\ifLyXsNoCenter
\LyXsNoCenterfalse
\newcommand{\noslidecentering}{
   \LyXsNoCentertrue%
}
\newcommand{\slidecentering}{
   \LyXsNoCenterfalse%
}
\newcommand{\lyxendslide}[1]{
   \ifLyXsNoCenter%
        \vfill%
   \fi%
   \ifcase \value{slidetype}%
        \or % no action for 0
        \end{slide} \or%
        \end{overlay} \or%
        \end{note}%
   \fi%
   \setcounter{slidetype}{0}
      \visible
}
\AtEndDocument{\lyxendslide{.}}
\newcommand{\lyxnewslide}[1]{
 \lyxendslide{.}
 \setcounter{slidetype}{1}
 \begin{slide}
}
\newenvironment{lyxlist}[1]
{\begin{list}{}
{\settowidth{\labelwidth}{#1}
 \setlength{\leftmargin}{\labelwidth}
 \addtolength{\leftmargin}{\labelsep}
 \renewcommand{\makelabel}[1]{##1\hfil}}}
{\end{list}}

%%%%%%%%%%%%%%%%%%%%%%%%%%%%%% User specified LaTeX commands.
% Uncomment to print out only slides and overlays
%
%\onlyslides{\slides}

% Uncomment to print out only notes
%
%\onlynotes{\notes}

\makeatother

\usepackage{babel}
\begin{document}
\global\long\def\GVWl{<_{\text{H}}}


\global\long\def\GVWg{>_{\text{\textnormal{H}}}}


\global\long\def\eqdef{\overset{\mathrm{_{def}}}{=}}


\global\long\def\equivdef{\overset{\mathrm{_{def}}}{\Leftrightarrow}}
	

\global\long\def\Sig{\mathcal{S}}


\global\long\def\HM{\mathrm{HM}}


\global\long\def\HC{\mathrm{HC}}


\global\long\def\LCM{\mathrm{LCM}}


\global\long\def\totaldeg{\mathrm{deg}}


\global\long\def\poly{\mathrm{poly}}


\global\long\def\sigidx{\mathrm{index}}


\setlength{\parskip}{5pt}


\lyxnewslide{}

Алгоритмы вычисления базиса Грёбнера идеала $\{f_{1},\dots,f_{m}\}$

Алгоритм Бухбергера -- последовательные редукциях S-пар. Много редукций
к нулю

Алгоритмы основанные на сигнатурах -- входным многочленам $f_{i}$
ставится в соответствие сигнатура $S(f_{i})=\left(1,i\right)$
\begin{itemize}
\item Задаётся согласованный с мономиальным $<$ порядок $\prec$
\item Распространяется на их полиномиальные комбинации

\begin{itemize}
\item Умножение на моном: $S(p)=(m,i)\Longrightarrow S(m'p)=(m'm,i)$
\item Сумма $S(p_{1}+p_{2})=\max_{\prec}(S(p_{1}),S(p_{2}))$
\end{itemize}
\end{itemize}

\lyxnewslide{}

Представление $p=\sum_{k}m_{k}\cdot g_{i_{k}}$ над $G\ni g_{i_{k}}$
\begin{itemize}
\item называется \emph{степенным}, если $\forall k\, HM(m_{k}g_{i_{k}})\preccurlyeq HM(p)$.

\begin{itemize}
\item Базис Грёбнера: любой элемент идеала имеет степенное представление
над ним
\end{itemize}
\item называется \emph{сигнатурным}, если $\forall k\, S(m_{k}g_{i_{k}})\preccurlyeq S(p)$.

\begin{itemize}
\item S-базис Грёбнера: любой элемент идеала имеет степенное и одновременно
сигнатурное представление над ним
\end{itemize}
\end{itemize}

\lyxnewslide{}
\begin{itemize}
\item Подмножество базиса Грёбнера $\left\{ g\in G|\totaldeg(g)\le d\right\} $
-- $d$-базис.

\begin{itemize}
\item Может инкрементально вычисляться для однородных многочленов
\end{itemize}
\item Подмножество S-базиса Грёбнера $\left\{ g\in G|S(g)\preccurlyeq\sigma\right\} $
-- $S_{\sigma}$-базис.

\begin{itemize}
\item Может инкрементально вычисляться
\end{itemize}
\end{itemize}

\lyxnewslide{}

Не сигнатурные алгоритмы
\begin{itemize}
\item Алгоритм Бухбергера (Бухбергер, 1976)
\item Алгоритм F4 (J.-C. Faugère, 1999)
\end{itemize}
Вычисляют обычный базис Грёбнера
\begin{itemize}
\item Применяются различные критерии, как правило формируемые в терминах
S-пар
\item Доказательство корректности основано на критерии редуцируемости всех
S-пар

\begin{itemize}
\item Каждый из критериев обосновывается независимо
\end{itemize}
\item Имеют редукции к нулю, поскольку никак не учитывают строение множества
сизигий
\end{itemize}

\lyxnewslide{}

Сигнатурные алгоритмы
\begin{itemize}
\item Вычисляют S-базис
\end{itemize}
\begin{tabular}{|c|c|c|}
\hline 
 & Отбрасывание мн-ов & Минимальный S-базис\tabularnewline
\hline 
\hline 
S-пары & F5 (J.-C. Faugère, 2002) & AP(A.Arri, J.Perry,2011)\tabularnewline
\hline 
без & G2V(2010) & TRB-MJ (L.Huang, 2010)\tabularnewline
 S-пар & GVW(2010) & SB (B.Roune, M.Stillman, 2012)\tabularnewline
\hline 
\end{tabular}
\begin{itemize}
\item Для регулярных входных данных, вычисляющие минимальный S-базис алгоритмы,
не отбрасывают никаких вычисленных многочленов и не имеют редукций
к нулю
\item Эффективность подтверждена экспериментами
\item Могут не использовать S-пары в формулировке
\item Инкрементально вычисляют $S_{\sigma}$-базисы
\end{itemize}

\lyxnewslide{}

Остановка и корректность сигнатурных алгоритмов
\begin{itemize}
\item в виде ``Алгоритма Бухбергера с особыми правилами отброса S-пар''
\item для алгоритма F5 не дано доказательства остановки
\item для других сложное доказательство, основанное на критерии редуцируемости
всех S-пар
\end{itemize}

\lyxnewslide{}

Важное для предлагаемых идей определение сравнения отмеченных многочленов:
сравнение старшего монома ``делённого'' на сигнатуру:

\[
h_{1}=(\sigma_{1},p_{1})\GVWl h_{2}=(\sigma_{2},p_{2})\Longleftrightarrow\HM(p_{1})\sigma_{2}\prec\HM(p_{2})\sigma_{1},
\]


При сигнатурной редукции:
\begin{itemize}
\item редуктор $>_{\mbox{H}}$ редуцируемого
\item результат редукции $<_{\mbox{H}}$ редуцируемого
\end{itemize}

\lyxnewslide{}

Доказано: F5 останавливается
\begin{enumerate}
\item Цепь отмеченных многочленов: $\left\{ h_{i}\right\} ,\Sig(h_{i-1})|\Sig(h_{i})$
\item Если не останавливается - существует бесконечная $<_{\mbox{H}}$-убывающая
цепь
\item В бесконечной цепи найдутся $\HM(h_{i})|\HM(h_{j})$
\item Это невозможно без учёта критериев из F5
\item Если есть сигнатурный редуктор, то есть и сигнатурный редуктор, удовлетворяющий
критериям
\end{enumerate}

\lyxnewslide{}

Алгоритм\texttt{ SingleStepSignatureGroebner}
\begin{lyxlist}{00.00.0000}
\item [{Вход:}] многочлены $\left\{ f_{1},\ldots,f_{m}\right\} $ с сигнатурами.
\item [{Переменные:}] $R$ -- промежуточный $S_{\sigma}$-базис, включающий
сизигии и $B$ -- многочлены ожидающие анализа
\item [{Результат:}] ненулевые элементы $R$ -- S-базис Грёбнера идеала
$I=\left(f_{1},\dots,f_{m}\right)$
\end{lyxlist}
Простое доказательство, не используещее S-пары


\lyxnewslide{}

Инициализация:

$R\leftarrow\{\mbox{известные сизигии}\}$

$B\leftarrow\{\mbox{входные многочлены}\}$


\lyxnewslide{}

\textbf{do while $B\neq\varnothing$:}
\begin{enumerate}
\item $(\sigma,p')\leftarrow$ элемент $B$ с $\prec$-минимальной сигнатурой
\item $B\leftarrow B\setminus\{b\in B,\Sig(b)=\sigma\}$
\item $p\leftarrow$Сигнатурно редуцировать $\left(\sigma,p'\right)$ по
$R$
\item $R\leftarrow R\cup\left\{ \left(\sigma,p\right)\right\} $
\item \textbf{if} $p\not=0$\textbf{:}

\begin{enumerate}
\item \textbf{for}$\{r\in R|0\ne r\GVWl\left(\sigma,p\right)\}$\textbf{:}$B\leftarrow B\cup\{\frac{\LCM(\HM(r),\HM(p))}{\HM(r)}r\}$\textbf{{[}Б{]}}
\item \textbf{for}$\{r\in R|r\GVWg\left(\sigma,p\right)\}$\textbf{:}$B\leftarrow B\cup\{\frac{\LCM(\HM(r),\HM(p))}{\HM(p)}\left(\sigma,p\right)\}$\textbf{{[}Б{]}}
\end{enumerate}
\item $B\leftarrow B\setminus\{b\in B|\exists r\in R\, r\GVWl b\mbox{ и есть делимость }\Sig(r)|\Sig(b)\}$\textbf{{[}Б{]}}
\end{enumerate}
\textbf{{[}Б{]} }-- отличия от Бухбергера, помимо добавления сигнатурности


\lyxnewslide{}

Доказательство остановки
\begin{enumerate}
\item Любой отмеченный многочлен из $B$, моном сигнатуры которого не равен
1, не редуцируется по $R$ сигнатурно
\item До редукции многочлена $p'$, сигнатуры из $\left\{ r\in R\,|\, r\GVWl(\sigma,p')\right\} $
не делят $\sigma$
\item После редукции многочлена $p'$ до $p$, старшие мономы из $\left\{ r\in R\,|\, r\GVWg(\sigma,p)\right\} $
не делят $\HM(p)$
\item После редукции многочлена $p'$ до $p$, элементы $R$ не могут одновременно
иметь старшие мономы, делящие $\HM(p)$, и сигнатуры, делящие $\sigma$
\item По лемме Диксона число пополнений $R$ конечно
\end{enumerate}

\lyxnewslide{}

Доказательство корректности:
\begin{enumerate}
\item Пусть $\sigma\succ0,R=\left\{ r_{i}\right\} $ -- S\emph{$_{\sigma}$}-базис
и выбраны $h_{1},h_{2},\Sig(h_{i})=\sigma$, которые нельзя сигнатурно
редуцировать по $R$. Тогда $\HM(h_{1})=\HM(h_{2})$ и у $h_{1}$
есть сигнатурное представление над $R\cup\left\{ h_{2}\right\} $.
\item На каждой итерации после шага 6 выполнен инвариант: для $\forall\sigma\prec$
сигнатур элементов $B$, найдутся $r_{\sigma}\in R,t_{\sigma}\in\mathbb{T}:\Sig(t_{\sigma}r_{\sigma})=\sigma$
и $t_{\sigma}r_{\sigma}$ не редуцируется сигнатурно по $R$.
\item На каждой итерации после шага 6 выполнен инвариант: $\forall h,\Sig(h)\prec$
сигнатур элементов $B$, имеет сигнатурное представление над $R$.
\end{enumerate}

\lyxnewslide{}

Алгоритмы вычисления базиса Грёбнера над $\mathbb{R}$ и $\mathbb{C}$
\begin{itemize}
\item Символические вычисления (исчерпывающий базис) -- рост символических
коэффициентов
\item Вычисления над $\mathbb{Z}$ и $\mathbb{Q}$ -- рост длины численных
коэффициентов
\item Вычисления с оценкой точности -- проблемы нулей

\begin{itemize}
\item Определение численными методами
\item Определение модулярными методами
\item Изменение порядка на мономах
\end{itemize}
\end{itemize}

\lyxnewslide{}

Приближённые числа

\[
(a,\varepsilon),a\in\mathbb{C},\varepsilon\in\mathbb{R}
\]


Специализации

\[
a_{0}\in\mathbb{C}\left|a_{0}-a\right|<\varepsilon
\]


Формализация задачи: ищем множество, содержащее решения при любых
специализациях.


\lyxnewslide{}
\begin{lyxlist}{00.00.0000}
\item [{Сложение}] $(a_{1},\varepsilon_{1})+(a_{2},\varepsilon_{2})=(a_{1}+a_{2},\varepsilon_{1}+\varepsilon_{2})$
\item [{Умножение}] $(a_{1},\varepsilon_{1})\times(a_{2},\varepsilon_{2})=(a_{1}a_{2},\varepsilon_{1}\left|a_{2}\right|+\varepsilon_{2}\left|a_{1}\right|+\varepsilon_{1}\varepsilon_{2})$
\item [{Вычитание}] на основе сложения и умножения на -1:
\[
(a_{1},\varepsilon_{1})-(a_{2},\varepsilon_{2})=(a_{1}-a_{2},\varepsilon_{1}+\varepsilon_{2})
\]

\item [{Обращение}] определяется только для приближённых чисел, для которых
0 не является специализацией, что эквивалентно $\left|a\right|-\varepsilon>0$:
\end{lyxlist}
\[
\frac{1}{(a,\varepsilon)}=\left(\frac{1}{a},\frac{\varepsilon}{\left|a\right|\left(\left|a\right|-\varepsilon\right)}\right)
\]



\lyxnewslide{}

Классификация необратимых приближённых элементов:
\begin{itemize}
\item \emph{символический ноль} -- при любой специализации входных данных
соответствующую вычисления дают точный ноль. 
\item \emph{ноль, индуцированный входными данными -- }некоторые, но не все
специализации дают точный ноль
\item \emph{ноль, внесённый вычислениями }-- не существует специализации,
дающей точный ноль
\end{itemize}

\lyxnewslide{}

Пример символического нуля:

\selectlanguage{english}%
\textrm{\normalsize 
\[
\begin{array}{ccc}
\poly(f_{1})= & y^{2}z+a, & \Sig(f_{1})=\left(1,1\right)\\
\poly(f_{2})= & y^{2}z^{2}+xz+1, & \Sig(f_{2})=\left(1,2\right)\\
\poly(f_{3})= & y^{3}z+xy+1, & \Sig(f_{3})=\left(1,3\right)\\
\poly(f_{4})= & \poly(f_{1})=y^{2}z+a, & \Sig(f_{4})=\left(1,1\right)\\
\poly(f_{5})= & \poly(f_{2})-z\poly(f_{4})=xz-az+1, & \Sig(f_{5})=\left(1,2\right)\\
\poly(f_{6})= & \poly(f_{3})-y\poly(f_{4})=xy-ay+1, & \Sig(f_{6})=\left(1,3\right)\\
\poly(f_{7})= & z\poly(f_{6})-y\poly(f_{5})=\left(a-a\right)yz+z-y, & \Sig(f_{7})=\left(z,3\right)
\end{array}
\]
}{\normalsize \par}

\selectlanguage{russian}%

\lyxnewslide{}

Пример индуцированного или внесённого нуля:

\[
\begin{array}{ccc}
\poly(f_{1})= & y^{2}z+z^{2}+az, & \Sig(f_{1})=\left(1,1\right)\\
\poly(f_{2})= & xyz, & \Sig(f_{2})=\left(1,2\right)\\
\poly(f_{3})= & xy^{2}+bx+1, & \Sig(f_{3})=\left(1,3\right)\\
\poly(f_{4})= & \poly(f_{1})=y^{2}z+z^{2}+az, & \Sig(f_{4})=\left(1,1\right)\\
\poly(f_{5})= & \poly(f_{2})=xyz, & \Sig(f_{5})=\left(1,2\right)\\
\poly(f_{6})= & \poly(f_{3})=xy^{2}+bx+1, & \Sig(f_{6})=\left(1,3\right)\\
\poly(f_{7})= & -\left(y\poly(f_{5})-x\poly(f_{4})\right)=xz^{2}+axz, & \Sig(f_{7})=\left(y,2\right)
\end{array}
\]
 
\[
\poly(f_{8})=(z\poly(f_{6})-x\poly(f_{4}))+\poly(f_{7})=
\]
\[
=\left(\left(xy^{2}z+bxz+z\right)-\left(xy^{2}z+xz^{2}+axz\right)\right)+xz^{2}+axz=((b-a)+a)xz+z,
\]
\[
\Sig(f_{8})=\left(z,3\right)
\]



\lyxnewslide{}

Сигнатурные алгоритмы для $\mathbb{R}$ и $\mathbb{C}$
\begin{itemize}
\item F5+TSV (J.-C. Faugère, Y. Liang, 2011): Комбинирование идей F5 и приближённых
вычислений с введением дополнительных многочленов $x_{1}^{i_{1}}\cdots x_{m}^{i_{m}}-\varepsilon$
для ``изменения'' порядка. Недостаток -- полный перезапуск алгоритмов
после добавления многочленов
\end{itemize}

\lyxnewslide{}

Порядок Term-Over-Pos на сигнатурах:

\[
\left(t_{1},i_{1}\right)\prec\left(t_{2},i_{2}\right)\Longleftrightarrow\left[\begin{aligned} & t_{1}<t_{2}\\
 & t_{1}=t_{2},i{}_{1}<i_{2}
\end{aligned}
\right..
\]
Порядок Pos-Over-Term на сигнатурах:

\[
\left(t_{1},i_{1}\right)\prec\left(t_{2},i_{2}\right)\Longleftrightarrow\left[\begin{aligned} & i{}_{1}<i_{2}\\
 & i{}_{1}=i_{2},t_{1}<t_{2}
\end{aligned}
\right..
\]


Порядок Schreyer$\prec$ на сигнатурах:

\[
\left(t_{1},i_{1}\right)\prec\left(t_{2},i_{2}\right)\Longleftrightarrow\left[\begin{aligned} & t_{1}HM(f_{i_{1}})<t_{2}HM(f_{i_{2}})\\
 & t_{1}HM(f_{i_{1}})=t_{2}HM(f_{i_{2}}),i{}_{1}<i_{2}
\end{aligned}
\right..
\]



\lyxnewslide{}

Взвешенный порядок $\prec_{w}$ на сигнатурах с $m$ входными многочленами
с параметром $w=(w_{1},\dots,w_{m})\in\mathbb{T}^{m}$:

\[
\left(t_{1},i_{1}\right)\prec_{w}\left(t_{2},i_{2}\right)\Longleftrightarrow\left[\begin{aligned} & t_{1}w_{i_{1}}<t_{2}w_{i_{2}}\\
 & t_{1}w_{i_{1}}=t_{2}w_{i_{2}},i{}_{1}<i_{2}
\end{aligned}
\right..
\]


Строится алгоритм \texttt{SingleStepSignatureGroebner} для $\mathbb{C}$
\begin{itemize}
\item Взвешенный порядок на сигнатурах позволяет не производить перезапуск
алгоритмов при добавлении многочлена $x_{1}^{i_{1}}\cdots x_{m}^{i_{m}}-\varepsilon$,
задавая ему подходящий вес
\item Модулярные вычисления для классификации нулей
\end{itemize}

\lyxnewslide{}

Модулярные методы носят вероятностный характер. Ложные нули получаются,
если выбранные значения оказываются корнем символического многочлена.
Это возможно, если:
\begin{itemize}
\item $\mathbb{Q}$-многочлен тождественно обнуляется по выбранному модулю

\begin{itemize}
\item Оценка на максимальный коэффициент для определения максимального числа
простых чисел, на которые делится
\end{itemize}
\item модулярная специализация $\mathbb{Q}$-многочлена имеет корень

\begin{itemize}
\item Оценка на максимальную степень
\end{itemize}
\end{itemize}

\lyxnewslide{}

Результаты:
\begin{itemize}
\item Доказана остановка F5
\item Прямое доказательство \texttt{SingleStepSignatureGroebner} без S-пар
\item Классификация необратимых приближённых элементов
\item \texttt{SingleStepSignatureGroebner для }$\mathbb{C}$\end{itemize}

\end{document}
