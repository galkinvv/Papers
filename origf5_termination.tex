%% LyX 2.0.2 created this file.  For more info, see http://www.lyx.org/.
%% Do not edit unless you really know what you are doing.
\documentclass[oneside,english]{amsart}
\usepackage[T2A]{fontenc}
\usepackage[latin9]{inputenc}
\usepackage{listings}
\usepackage[a4paper]{geometry}
\geometry{verbose}
\usepackage{color}
\usepackage{babel}
\usepackage{amsthm}
\usepackage{amssymb}
\usepackage[unicode=true,pdfusetitle,
 bookmarks=true,bookmarksnumbered=false,bookmarksopen=true,bookmarksopenlevel=1,
 breaklinks=false,pdfborder={0 0 1},backref=false,colorlinks=true]
 {hyperref}

\makeatletter
%%%%%%%%%%%%%%%%%%%%%%%%%%%%%% Textclass specific LaTeX commands.
\numberwithin{equation}{section}
\numberwithin{figure}{section}
\newcommand{\lyxrightaddress}[1]{
\par {\raggedleft \begin{tabular}{l}\ignorespaces
#1
\end{tabular}
\vspace{1.4em}
\par}
}
\theoremstyle{plain}
\newtheorem{thm}{\protect\theoremname}
  \theoremstyle{remark}
  \newtheorem{claim}[thm]{\protect\claimname}
  \theoremstyle{definition}
  \newtheorem{defn}[thm]{\protect\definitionname}
  \theoremstyle{plain}
  \newtheorem{fact}[thm]{\protect\factname}
  \theoremstyle{definition}
  \newtheorem{example}[thm]{\protect\examplename}
  \theoremstyle{plain}
  \newtheorem{lem}[thm]{\protect\lemmaname}
  \theoremstyle{plain}
  \newtheorem{cor}[thm]{\protect\corollaryname}

%%%%%%%%%%%%%%%%%%%%%%%%%%%%%% User specified LaTeX commands.

\newcommand{\Sig}{\mathcal{S}}
\newcommand{\HM}{\mathrm{HM}}
\newcommand{\totaldeg}{\mathrm{deg}}

\makeatother

  \providecommand{\claimname}{Claim}
  \providecommand{\corollaryname}{Corollary}
  \providecommand{\definitionname}{Definition}
  \providecommand{\examplename}{Example}
  \providecommand{\factname}{Fact}
  \providecommand{\lemmaname}{Lemma}
\providecommand{\theoremname}{Theorem}

\begin{document}

\title{Termination of original F5}


\author{Vasily Galkin}

\maketitle

\lyxrightaddress{Moscow State University}
\begin{abstract}
The original F$_{5}$ algorithm described in \cite{F5-Orig} is formulated
for any homogeneous polynomial set input. The correctness of output
is shown for any input that terminates the algorithm, but the termination
itself is proved only for the case of input being regular polynomial
sequence. This article shows that algorithm correctly terminates for
any homogeneous input without any reference to regularity. The scheme
contains two steps: first it is shown that if the algorithm does not
terminate it eventually generates two polynomials where first is a
reductor for the second. But first step doesn't show that this reduction
is permitted by criteria introduced in F$_{5}$. The second step shows
that if such pair exists then there exists another pair for which
the reduction is permitted by all criteria. Existence of such pair
leads to contradiction.
\end{abstract}

\section{Introduction}

The Faug�re's F$_{5}$ algorithm is known to be efficient method for
Gr�bner basis computation but one of the main problems with it's practical
usage is lack of termination proof for all cases. The original paper
\cite{F5-Orig} and detailed investigations in \cite{F5-Revisited}
states the termination for the case of reductions to zero absence,
which practically means termination proof for the case of input being
regular polynomial sequence. But for most input sequences the regularity
is not known, so this is not enough for practical implementations
termination proof. One of the approaches to solve this issue is adding
of additional checks and criteria for ensuring algorithm termination.
Examples are \cite{term-mod-1,term-mod-2,Modifying-for-termination}.

This paper introduces another approach for termination proof of original
algorithm without any modifications. The first step of proof is based
on the idea of S-pair chains which are introduced in this paper. The
second step of proof is based on the method described in Theorem 21
of \cite{F5C} for the proof of F5C algorithm correctness: the representation
of an S-polynomial as the sum of multiplied polynomials from set computed
by F5C can be iteratively rewritten using replacements for S-pairs
and rejected S-pair parts until a representation with certain good
properties is achieved after finitely many steps.

This article shows that the hypothesis of this method can be weakened
to apply it for the set at any middle stage of F5 computations and
the conclusions can be strengthened to use them for termination proof.
The paper is designed as alternative termination proof for exact algorithm
described in \cite{F5-Orig}, so the reader is assumed to be familiar
with it and all terminology with names for algorithm steps are borrowed
from there.


\section{Possibilities for infinite cycles in F5}


\subsection{Inside AlgorithmF5: $d$ growth}

This section aims to prove the following
\begin{claim}
\label{iterations-d_grow}If the number of \textbf{while} cycle iteration
inside AlgorithmF5 is infinite then the $d$ value infinitely grow.\end{claim}
\begin{proof}
Let's suppose that there is an input $\left\{ f_{1},\ldots,f_{m}\right\} $
over $\mathcal{K}[x_{1},\ldots,x_{n}]$ for which original F$_{5}$
does not terminate, and that it is shortest input of such kind --
the algorithm do terminate on shorter input $\left\{ f_{2},\ldots,f_{m}\right\} $.
This means that last iteration of outer cycle in incrementalF5 does
not terminate, so some call to AlgorithmF5 does not terminate. To
investigate this we need to study how the total degree $d$ can change
during execution of the cycle inside AlgorithmF5. Let's call $d_{j}$
the value of $d$ on $j$-th cycle iteration and extend it to $d_{0}=-1$.
The simple property of $d_{j}$ is it's non-strict growth: $d_{j}\geqslant d_{j-1}$.
It holds because on the $j-1$-th iteration all polynomials in $R_{d}$
have degree $d_{j-1}$ and therefore all new generated critical pairs
have degree at least $d_{j-1}$. Now suppose that $j$ is number of
some fixed iteration. At the iteration $j$ all critical pairs with
degree $d_{j}$ are extracted from $P$. After call to Reduction some
new critical pairs are added to $P$. There exist a possibility that
some of them has degree $d_{j}$. We're going to show that all such
critical pairs do not generate S-polynomials in the next iteration
of algorithm because they are discarded.

For each new critical pair $[t,u_{1},r_{1},u_{2,}r_{2}]$ generated
during iteration $j$ at least one of the generating polynomials belong
to $R_{d}$ and no more than one belong to $G_{i}$ at the beginning
of the iteration. All polynomials in $R_{d}$ are generated by Reduction
function by appending single polynomials to $Done$. So we can select
from one or two $R_{d}$-belonging generators of critical pair a polynomial
$r_{k}$ that was added to $Done$ later. Then we can state that the
other S-pair part $r_{3-k}$ was already present in $G\cup Done$
at the moment of $r_{k}$ was added to $Done$. So the TopReduction
tries to reduce $r_{k}$ by $r_{3-k}$ but failed to do this because
one of IsReducible checks (a) - (d) forbids this.

From the other hand for critical pairs with degree equal to $d_{j}$
we have $u_{k}=1$ because total degree of critical pair is equal
to total degree of it's generator $r_{k}$. This means that value
$u_{3-k}$ is equal to $\frac{\HM(r_{k})}{\HM(r_{3-k})}$ so the IsReducible's
rule (a) allow reduction $r_{k}$ by $r_{3-k}$. It follows that only
checks (b) - (d) are left as possibilities.

Suppose that reduction was forbidden by (b). This means that there
is a polynomial in $G_{i+1}$ that reduces $u_{3-k}\Sig(r_{3-k})$.
For our case it means that in the CritPair function the same check
$\varphi(u_{3-k}\Sig(r_{3-k}))=u_{3-k}\Sig(r_{3-k})$ fails and such
critical pair wouldn't be created at all. So the rule (b) can't forbid
reduction too.

Suppose that reduction was forbidden by (c). This means that there
is a rewriting for the multiplied reductor. So for our case it means
that Rewritten?$\left(u_{3-k},r_{3-k}\right)$ returns true at the
moment of TopReduction execution, so it still returns true for all
algorithm execution after this moment because rewritings do not disappear.

Suppose that reduction was forbidden by (d). The pseudo code in \cite{F5-Orig}
is a bit unclear at this point, but the source code of procedure FindReductor
attached to \cite{F5-Revisited} is more clear and states that the
reductor is discarded if both monomial of signature and index of signature
are equal to those of polynomial we're reducing (it's signature monomial
is r{[}k0{]}{[}1{]} and index is r{[}k0{]}{[}2{]} in the code):

\begin{lstlisting}
if (ut eq r[k0][1]) and (r[j][2] eq r[k0][2]) then
	// discard reductor by criterion (d)
	continue;
end if;
\end{lstlisting}
For our case it means that signatures of $r_{k}$ and $u_{3-k}r_{3-k}$
are equal. This leads to fact that Rewritten?$\left(u_{3-k},r_{3-k}\right)$
returns true after adding rule corresponding to $r_{k}$ because $u_{3-k}\cdot r_{3-k}$
is rewritable by $1\cdot r_{k}$. So like in case (c) Rewritten?$\left(u_{3-k},r_{3-k}\right)$
returns true at the moment of TopReduction execution.

Now consider Spol function execution for some S-pair with total degree
$d_{j}$ generated during iteration $j$. It executes in $j+1$ iteration
of AlgorithmF5 cycle which is far after TopReduction execution for
$r_{k}$ in algorithm flow so for both cases (c) and (d) call to Rewritten?$\left(u_{3-k},r_{3-k}\right)$
inside Spol returns true. It means that at the $j+1$ step no S-pair
with total degree $d_{j}$ can add polynomial to $F$.

In the conclusion we have: 
\begin{itemize}
\item the first possibility of $d_{j+1}$ and $d_{j}$ comparison is $d_{j+1}=d_{j}$.
In this case $F$ is empty on $j+1$ iteration and therefore $P$
does not contain any pairs with degree $d_{j}$ after $j+1$ iteration's
finish. So $d_{j+2}>d_{j+1}$. 
\item the other possibility is $d_{j+1}>d_{j}$.
\end{itemize}
In conjunction with non-strict growth this gives $\forall j\,\, d_{j+2}>d_{j}$
which proves the claim \ref{iterations-d_grow}.
\end{proof}

\subsection{Inside Reduction: $\Sig(h)$ growth}

This section aims to prove the following
\begin{claim}
\label{Every_cycle_iteration_finish}Every cycle iteration inside
AlgorithmF5 does terminate, in particular all calls to Reduction terminate.\end{claim}
\begin{proof}
Firstly we need to get some facts about signatures of polynomials
in $ToDo$ set inside $j$-th iteration of AlgorithmF5. All the critical
pairs created with CritPair inside the AlgorithmF5 called for new
input sequence item $f_{i}$ have greater S-pair part with signature
index $i$. So, all polynomial initially forming $ToDo$ are generated
by Spol and have signature index $i$. The polynomials added in $ToDo$
inside Reduction are generated by TopReduction. Each such polynomial
has signature $u\Sig(r_{k_{1}})$ that is greater than signature $\Sig(r_{k_{0}})$
of some other polynomial that already is $ToDo$ element because the
case $u\Sig(r_{k_{1}})=\Sig(r_{k_{0}})$ is discarded inside IsReducible
(d) and the case $u\Sig(r_{k_{1}})\prec\Sig(r_{k_{0}})$ is an execution
path inside TopReduction that does not add new polynomials in $ToDo$.
So signature index of all added elements are equal to $i$ too. From
the other hand all polynomials inside $ToDo$ has the same total degree
$d_{j}$. This property together with index equality shows that the
total degree of signature monomials is equal to the $d_{j}-\totaldeg(f_{i})$
for all $ToDo$ elements.
\begin{defn}
The reduction of labeled polynomial $r_{k}$ with a labeled polynomial
$r_{m}$ is called \emph{signature-safe} if $\Sig(r_{k})\succ t\cdot\Sig(r_{m})$,
where $t=\frac{\HM(r_{k})}{\HM(r_{m})}$ is monomial multiplier of
reductor. The reductor corresponding to signature-safe reduction is
called \emph{signature-safe reductor}.
\end{defn}
We want to show that only possible algorithm non-termination situation
correspond to the case of infinite $d_{j}$ growth. We showed that
non-termination leads to AlgorithmF5 does not return, and that it
can't stuck in iterations with same $d$ value. So, the only possibilities
left are infinite $d$ growth and sticking inside some iteration.
We are going to show that such sticking is not possible. The AlgorithmF5
contains 3 cycles:
\begin{itemize}
\item \textbf{for} cycle inside Spol does terminate because it's number
of iterations is limited by a count of critical pairs which is fixed
at cycle beginning
\item \textbf{for} cycle inside AlgorithmF5 iterating over $R_{d}$ elements
also does terminate because count of $R_{d}$ elements is fixed at
cycle beginning
\item the most complex case is the \textbf{while} cycle inside Reduction
which iterates until $ToDo$ becomes empty. The $ToDo$ set is extended
by new elements during TopReduction execution, but all new elements
have signatures greater than $\Sig(h)$ -- signature of smallest $ToDo$
element at the beginning of cycle iteration. This leads to non-strict
growth of $\Sig(h)$ during cycle iterations. Now the Reduction termination
is got by showing two facts:

\begin{itemize}
\item There is finite number of iterations with fixed smallest $ToDo$ signature
$\Sig(h)=S_{0}$. Every step of a cycle inside Reduction performs
an operation with a polynomial of signature $S_{0}$ -- let's call
them $S_{0}$-signature polynomials. The number of $S_{0}$-signature
polynomials in $ToDo$ does not increase after first $S_{0}$-signature
polynomial was taken as $h$ because all polynomials TopReduction
adds in $ToDo$ has signatures greater than smallest. The real presence
of example input that produces more than one polynomial in $ToDo$
with same signature $S_{0}$ is left as open question here. It is
not needed to termination proof - only the count finiteness is crucial.
The performed operation can be one of the following three operations
and we are going to show that neither of them can be performed infinite
number of times:

\begin{itemize}
\item Some polynomial with a signature $S_{0}$ is transferred from $ToDo$
to $Done$. This can be performed only finite number of times because
the count of polynomials with signature $S_{0}$ inside $ToDo$ is
finite.
\item Some polynomial is signature-safe top-reduced. This can be done only
finite number of steps for each of polynomials because each top-reduction
$\prec$-reduces the head monomial and the set of monomials is well-ordered
with $\prec$. Together with finiteness of $S_{0}$-signature polynomials
count this shows that the number of such steps is finite too.
\item Set of $S_{0}$-signature polynomials is not touched, but a possibility
of signature-unsafe top-reduction adds a polynomial in $ToDo$. The
consecutive number of operations of this type without interleaving
second type operations which change the $S_{0}$-signature polynomials
is limited. This holds because the number of all signature-safe and
signature-unsafe possible top-reductions for $S_{0}$-signature polynomials
is limited to number of such polynomials multiplied by number of possible
reducers. Therefore the fixed count of $S_{0}$-signature polynomials
gives the number of polynomials to reduce and the fixed $|G\cup Done|$
gives the count of reducers. Every (polynomial-to-reduce; reducer)
pair can produce no more than one new polynomial in $ToDo$ because
after the first addition the multiplied reducer will be rewritten
by added polynomial and IsReducible discards this pair. The limit
on consecutive number of operations of this type gives a limit on
overall number of iterations with $\Sig(h)=S_{0}$ because the finiteness
of other operation types was proved. 
\end{itemize}
\item All different signatures $\Sig(h)$ appeared during Reduction execution
have the same index and total degree. So their count is finite because
number of monomials with a fixed total degree is finite.
\end{itemize}
\end{itemize}
We got that all of the cycles inside AlgorithmF5 do finish and the
claim \ref{Every_cycle_iteration_finish} is proved.
\end{proof}
This gives the result about algorithm behavior for the non-terminated
case:
\begin{claim}
\label{d-does-grow}If the algorithm does not terminate for some input
then the value of $d$ infinitely grow during iterations.\end{claim}
\begin{proof}
Follows from combination of claims \ref{iterations-d_grow} and \ref{Every_cycle_iteration_finish}.
\end{proof}

\section{S-pair-chains}

The claim \ref{d-does-grow} shows that algorithm non termination
leads to existence of infinite sequence of nonzero labeled polynomials
being added to $G_{i}$ and the total degrees of polynomials in the
sequence infinitely grow. So, in this case algorithm generates an
infinite sequence of labeled polynomials $\left\{ r_{1},r_{2},\ldots,r_{m},\ldots,r_{l},\ldots\right\} $
where $r_{1},\ldots,r_{m}$ correspond to $m$ input polynomials and
other elements are generated either in Spol or in TopReduction. In
both cases new element $r_{l}$ is formed as S-polynomial of two already
existing polynomials already present in the list. We will write $l^{*}$
and $l_{*}$ for the indexes of the polynomials used to generate $l$-th
element and $\overline{u_{l}}$, $\underline{u_{l}}$ for monomials
they are multiplied. Note that $l^{*}$ correspond to the part with
greater signature: $poly(r_{l})=\overline{u_{l}}poly(r_{l^{*}})-\underline{u_{l}}poly(r_{l_{*}})$
and $\Sig(r_{l})=\overline{u_{l}}\Sig(r_{l^{*}})\succ\underline{u_{l}}\Sig(r_{l_{*}})$.
The $poly(r_{l})$ value can further change inside TopReduction to
the polynomial with a smaller HM, but the $\Sig(r_{l})$ does never
change after creation. Now, we want to select an infinite sub-sequence
$\left\{ r_{k_{1}},r_{k_{2}},\ldots,r_{k_{n}},\ldots\right\} $ in
that sequence with the property that $r_{k_{n}}$ is an S-polynomial
generated by $r_{k_{n-1}}=r_{k_{n}^{*}}$ and some other polynomial
corresponding to smaller by signature S-pair part, so $\Sig(r_{k_{n}})=\overline{u_{k_{n}}}\Sig(r_{k_{n-1}})$
and 
\begin{equation}
\Sig(r_{k_{n-1}})|\Sig(r_{k_{n}}).\label{eq:s-pair-chain-def}
\end{equation}

\begin{defn}
Finite or infinite labeled polynomial sequence which successive elements
satisfy property \ref{eq:s-pair-chain-def} will be called \emph{S-pair-chain}.
\end{defn}
Every generated labeled polynomial $r_{l}$ has an finite S-pair-chain
ending with that polynomial. This chain can be constructed in reverse
direction going from it's last element $r_{l}$ by selecting every
step from a given polynomial $r_{n}$ a polynomial $r_{n^{*}}$ which
was used to generate $r_{n}$ as $Spoly(r_{n^{*}},r_{n_{*}})$. The
resulting S-pair-chain has the form $\{r_{q},\ldots,r_{l^{**}},r_{l^{*}},r_{l}\}$
where all polynomials has the same signature index $q=index(r_{l})$
and the first element is the input polynomial of that index.

The first fact about S-pair-chains is based on the rewritten criteria
and consists in the following theorem.
\begin{thm}
Every labeled S-polynomial can participate as the first element only
in finite number of S-pair-chains of length 2.\end{thm}
\begin{proof}
The Algorithm F5 computes S-polynomials in 2 places: in procedure
SPol and in the procedure TopReduction. It's important that in both
places the Rewritten? check for the part of S-Poly with greater signature
is performed just before the S-polynomial is constructed. In the first
case the SPol is checking that itself, in the TopReduction the check
is in the IsReducible procedure. And in both cases the computed S-polynomial
is immediately added to the list of rules, as the newest element.
So, at the moment of the construction of S-polynomial with signature
$s$ we can assert that the higher part of S-pair correspond to the
newest rule with signature dividing $s$ -- this part even may be
determined by list of rules and $s$ without knowing anything other
about computation.

Consider arbitrary labeled polynomial $r_{L}$ with signature $\Sig(r_{L})=s$
and an ordered by generating time subset $\{r_{l_{1}},\ldots,r_{l_{i}},\ldots\}$
of labeled polynomials with signatures satisfying $\Sig(r_{l_{i}})=v_{i}\Sig(r_{L})$.
From the signature divisibility point of view all of the possibly
infinite number of pairs $\{r_{L},r_{l_{i}}\}$ can be S-pair-chains
of length 2. But the ideal $\left(v_{i}\right)$ in $T$ is finitely
generated by Dickson's lemma, so after some step $i_{0}$ we have
$\forall i>i_{0}\,\exists j\leqslant i_{0}$ such that $v_{j}|v_{i}$.
So $\forall i>i_{0}$ the sequence $\{r_{L},r_{l_{i}}\}$ is not S-pair-chain
because $\Sig(r_{L})\cdot u_{i}$ is rewritten by $\Sig(r_{l_{j}})\cdot\frac{v_{i}}{v_{j}}$
and no more than $i_{0}$ S-pair-chains of length 2 with first element
$r_{L}$ exist.\end{proof}
\begin{defn}
The finite set of ends of 2-length S-pair-chains starting with $r_{L}$
will be called \emph{S-pair-descendants} of $r_{L}$.\end{defn}
\begin{thm}
If the algorithm does not terminate for some input then there exists
infinite S-pair-chain $\{h_{i}\}$.\end{thm}
\begin{proof}
Some caution is required while dealing with infinities, so we give
the following definition. 
\begin{defn}
The labeled polynomial $r_{l}$ is called \emph{chain generator} if
there exist infinite number of different finite S-pair-chains starting
with $r_{l}$.
\end{defn}
If the algorithm doesn't terminate the input labeled polynomial $r_{1}=(f_{1},1F_{1})$
is chain generator because every labeled polynomial $r_{l}$ generated
in the last non-terminating call to Algorithm F5 has signature index
1 so there is an S-pair-chain $\{r_{1},\ldots,r_{l^{**}},r_{l^{*}},r_{l}\}$.

Now assume that some labeled polynomial $r_{l}$ is known to be a
chain generator. Then one of the finite number of S-pair-descendants
of $r_{l}$ need to be a chain generator too, because in the other
case the number of different chains of length greater than 2 coming
from $r_{l}$ was limited by a finite sum of the finite counts of
chains coming from every S-pair-descendant, and the finite number
of length 2 chains coming from $r_{l}$. So, if labeled polynomial
$r_{l}$ is chain-generator, we can select another chain generator
from it's S-pair-descendants. In a such way we can find infinite S-pair-chain
starting with $r_{1}$ and consisting of chain generators which proves
the theorem.
\end{proof}
For the next theorem we need to introduce monomial quotients order
by transitively extending the monomial ordering: $\frac{m_{1}}{m_{2}}>_{q}\frac{m_{3}}{m_{4}}\Leftrightarrow m_{1}m_{4}>m_{3}m_{2}$.
\begin{thm}
\label{thm:f_g_3_props}If the algorithm does not terminate for some
input then after some finite step the set $G$ contains a pair of
labeled polynomials $f',f$ with $f$ generated after $f'$ that satisfies
the following 3 properties:

\[
\HM(f')|\HM(f),
\]


\[
\frac{\HM(f')}{\Sig(f')}>_{q}\frac{\HM(f)}{\Sig(f)},
\]


\[
\Sig(f')|\Sig(f).
\]
\end{thm}
\begin{proof}
For working with S-pair-chains it is important that the polynomial
can never reduce after it was used for S-pair generation as higher
S-pair part. That's true because all polynomials that potentially
can reduce are stored in set $ToDo$, but all polynomials that are
used as higher S-pair part are stored in $G$ or in $Done$. So we
may state that the polynomial $h_{n}$ preceding polynomial $h_{n+1}$
in the S-pair-chain keeps the same $poly(h_{n})$ value after it was
used for some S-pair generation and we can state that
\[
poly(h_{n+1})=c\frac{\Sig(h_{n+1})}{\Sig(h_{n})}poly(h_{n})+g_{n},
\]
where $g_{n}$ is the polynomial corresponding to smaller part of
S-pair used to generate $h_{n+1}$ from $h_{n}$ and satisfy:
\begin{equation}
\HM(h_{n+1})<\HM\left(\frac{\Sig(h_{n+1})}{\Sig(h_{n})}h_{n}\right)=\HM(g_{n}),\,\Sig(h_{n+1})=\Sig\left(\frac{\Sig(h_{n+1})}{\Sig(h_{n})}h_{n}\right)\succ\Sig(g_{n}).\label{eq:spair-chain}
\end{equation}
From the first inequality in \ref{eq:spair-chain} we can get $\frac{\HM(h_{n})}{\Sig(h_{n})}>_{q}\frac{\HM(h_{n+1})}{\Sig(h_{n+1})}$,
so in the S-pair-chain the quotients $\frac{\HM(h_{i})}{\Sig(h_{i})}$
are strictly descending according to quotients ordering. This fact
can't be used directly to show chains finiteness because unlike the
ordering of monomials the ordering of monomial quotients is not well
ordering -- for example the sequence $\frac{x}{x}>_{q}\frac{x}{x^{2}}>_{q}\cdots>_{q}\frac{x}{x^{n}}>_{q}\cdots$
is infinitely decreasing.

There is two possible cases for relation between HMs of consecutive
elements. We have $\Sig(h_{n})|\Sig(h_{n+1})$, so they are either
equal or $\totaldeg(h_{n})<\totaldeg(h_{n+1})$. For the first case
$\HM(h_{n+1})<\HM(h_{n})$ with the equal total degrees for the other
case $\HM(h_{n+1})>\HM(h_{n})$ because total degrees are different.
So the sequence of infinite S-pair-chain HMs consists of blocks with
fixed total degrees where HMs inside a block are strictly decreasing.
Block lengths can be equal to one and the total degree of blocks are
increasing. This leads to the following properties: S-pair-chain $\{h_{i}\}$
can't contain elements with equal HMs and $\HM(h_{i})|\HM(h_{j})$
is possible only for $i<j$ and $\totaldeg(h_{i})<\totaldeg(h_{j})$.

This allows us to use technique analogous to Proposition 14 from \cite{Arri-Perry-Revised}:
consider HMs of infinite S-pair-chain $\{h_{i}\}$. They form an infinite
sequence in $T$, so by Dickson's lemma there exists 2 polynomials
in sequence with $\HM(h_{i})|\HM(h_{j})$. Therefore from the previous
paragraph we have $i<j$ and with the S-pair-chain properties we get
$\Sig(h_{i})|\Sig(h_{i+1})|\cdots|\Sig(h_{j})$ and $\frac{\HM(h_{i})}{\Sig(h_{i})}>_{q}\frac{\HM(h_{i+1})}{\Sig(h_{i+1})}>_{q}\cdots>_{q}\frac{\HM(h_{j})}{\Sig(h_{j})}$,
so we take $f'=h_{i}$ and $f=h_{j}$.
\end{proof}
The last property about signature division from the theorem claim
is the consequence of dealing with S-pair-chains and is not used in
the following. But the first to properties are used to construct a
signature-safe reductor.
\begin{fact}
If no polynomials are rejected by criteria checks (b) and (c) inside
IsReducible the algorithm does terminate.\end{fact}
\begin{proof}
The above proof of theorem\ref{thm:f_g_3_props} does not rely on
any correspondence between orderings on signatures and terms. But
the original algorithm F$_{5}$ uses the same ordering for both cases
and now we utilize this fact and make a transition from one to other
to get relation on signatures for polynomials from theorem \ref{thm:f_g_3_props}
claim: 
\[
\Sig(g)\succ t\cdot\Sig(f),\mbox{ where }t=\frac{\HM(g)}{\HM(f)}\in T.
\]
The last inequality with HM's division property from theorems result
shows that $tf$ can be used as a reducer for $g$ in TopReduction
from the signature point of view -- i.e. it satisfy checks (a) and
(d) inside IsReducible and it's signature is smaller. In the absence
of criteria checks (b) and (c) this would directly lead to contradiction
because at the time $g$ was added to the set $G$ labeled polynomial
$f$ already had been there, so the TopReduction should had reduced
$g$ by $f$.
\end{proof}
But the existence of criteria allow the situation in which $tf$ is
rejected by criteria checks in (b) or (c) inside IsReducible. The
idea is to show that even in this case there can be found another
possible reducer for $g$ that is not rejected and anyway lead to
contradiction and the following parts of paper aim to prove it.


\section{S-pairs with signatures smaller than $\Sig(g)$}

In this and following sections $g$ is treated as some fixed labeled
polynomial with signature index 1 added to $Done$ in some algorithm
iteration. Let us work with algorithm state just before adding $g$
to $Done$ during call to AlgorithmF5 with $i=1$. Consider a finite
set $G_{1}\cup Done$ of labeled polynomials at that moment. This
set contains positions of labeled polynomials in $R$, so it's elements
can be ordered according to position in $R$ and written as an ordered
integer sequence $G_{g}=\{b_{1},\ldots,b_{N}\}$ with $b_{j}<b_{j+1}$.
It should be noted that this order does correspond to the order of
labeled polynomials in the sequence produced by concatenated rule
arrays $Rule[m]:Rule[m-1]:\cdots:Rule[1]$ because addition of new
polynomial to $R$ is always followed by addition of corresponding
rule. But this order may differ from the order polynomials were added
to $G_{1}\cup Done$ because polynomials with same total degree are
added to $Done$ in the increasing signature order, while the addition
polynomials with same total degree to $R$ is performed in quite random
order inside Spol and TopReduction procedures. For the simplicity
we will be speaking about labeled polynomials $b_{j}$ from $G_{g}$,
assuming that $G_{g}$ is not the ordered positions list but the ordered
list of labeled polynomials themselves corresponding to those positions.
In this terminology we can say that all input polynomials $\left\{ f_{1},\ldots,f_{m}\right\} $
do present in $G_{g}$, because they all present in $G_{1}$ at the
moment of its creation.

S-pairs can be processed in a different ways inside the algorithm
but the main fact we need to know about their processing is encapsulated
in the following properties which correspond to the properties used
in Theorem 21 in \cite{F5C} but are taken during arbitrary algorithm
iteration without requirements of termination.
\begin{thm}
\label{thm:Exist-gg-repr}At the moment of adding $g$ to $Done$
every S-pair of $G_{g}$ elements which signature is smaller than
$\Sig(g)$ satisfies one of three properties:\end{thm}
\begin{enumerate}
\item S-pair has a part that is rejected by the normal form check $\varphi$
(in CritPair or in IsReducible). Such S-pairs will be referenced as
\emph{S-pairs with a part that satisfies F5 criterion}. 
\item S-pair has a part that is rejected by the Rewritten? check (in SPol
or in IsReducible). Such S-pairs will be referenced as \emph{S-pairs
with a part that satisfies Rewritten criterion}.
\item S-pair was not rejected, so it's S-polynomial was signature-safe reduced
by $G_{g}$ elements and the result is stored in $G_{g}$. Such S-pairs
will be referenced as \emph{S-pairs with a computed $G_{g}$-representation}.\end{enumerate}
\begin{proof}
S-pairs of $G_{g}$ elements are processed in two paths in the algorithm.
The main path is for S-pairs with total degree greater than total
degrees of polynomials generated it. Such S-pairs are processed in
the following order:
\begin{itemize}
\item in the AlgorithmF5 they are passed in CritPair function while moving
elements to $G_{i}$ from $R_{d}=Done$ or while processing input
polynomial $r_{i}$. 
\item The CritPair function either discards them by normal form check $\varphi$
or adds to $P$
\item The S-pair is taken from $P$ and passed to SPol function
\item The SPol function either discards them by Rewritten? check or adds
S-polynomial to $F=ToDo$
\item At some iteration the Reduction procedure takes S-polynomial from
$ToDo$, performs some signature-safe reductions and adds result to
$Done$.
\end{itemize}
The other processing path is for special S-pairs corresponding to
reductions forbidden by algorithm -- the case when S-pair is generated
by polynomials $r_{l^{*}}$ and $r_{l_{*}}$ such that $\HM(r_{l^{*}})|\HM(r_{l_{*}})$
so that S-polynomial has a form $\overline{u_{l}}\cdot poly(r_{l^{*}})-1\cdot poly(r_{l_{*}})$.
Such situation is possible for two $G_{g}$ elements if reduction
of $r_{l_{*}}$ by $r_{l^{*}}$ was forbidden by signature comparison
in TopReduction or by checks in IsReducible. So for this case the
path of S-pair ``processing'' is the following:
\begin{itemize}
\item The S-pair part $\overline{u_{l}}\cdot r_{l^{*}}$ is checked in IsReducible.
(a) is satisfied because $\HM(r_{l^{*}})|\HM(r_{l_{*}})$. It can
be rejected by one of the other checks:

\begin{itemize}
\item Rejection by check (b) correspond to the normal form check $\varphi$
for $\overline{u_{l}}\cdot r_{l^{*}}$
\item Rejection by check (c) correspond to the Rewritten? check for $\overline{u_{l}}\cdot r_{l^{*}}$
\item Rejection by check (d) means that either $\overline{u_{l}}\cdot r_{l^{*}}$
or $1\cdot r_{l_{*}}$ can be rewritten by other, so if S-pair was
not rejected by check (c) this type of rejection means that S-pair
part $1\cdot r_{i_{1}}$ fails to pass Rewritten? check.
\end{itemize}
\item The non-rejected in IsReducible S-pair is returned in the TopReduction.
Signature comparison in TopReduction forbids the reduction of $r_{l_{*}}$
by $r_{l^{*}}$ and returns computed S-polynomial corresponding to
the S-pair in the set $ToDo_{1}$
\item The Reduction procedure add this polynomial in $ToDo$
\item The last step is equal to for both processing paths: at some iteration
the Reduction procedure takes S-polynomial from $ToDo$, performs
some signature-safe reductions and adds result to $Done$
\end{itemize}
It can be seen that after S-pair processing termination every S-pair
is either reduced and added to $Done$ or one of its S-pair parts
is rejected by normal form $\varphi$ or Rewritten? check. Some S-pairs
can be processed by processing paths multiple times, for example this
is done in the second iteration inside Algorithm F5 with same $d$
value. But this doesn't matter because the start and finish of the
processing path one time guarantees that S-pair is rejected by Rewritten?
check in all other processing attempts.

The processing path is not a single procedure and for the case of
algorithm infinite cycling some S-pairs are always staying in the
middle of the path having S-pair queued in $P$ or S-polynomial in
$ToDo$. So we have to select S-pairs which processing is already
finished at the fixed moment we studying. The elements from $P$ and
$ToDo$ in AlgorithmF5 and Reduction procedures are taken in the order
corresponding to growth of their signatures. So S-pairs with signature
smaller than $\Sig(g)$ can be split in the following classes:
\begin{itemize}
\item S-pairs with signature $w$ such that $index(w)>index(\Sig(g))=1$.
They were processed on previous calls of AlgorithmF5.
\item S-pairs with signature $w$ such that $index(w)=index(\Sig(g))=1,\totaldeg(w)<\totaldeg(\Sig(g))$.
They were processed on previous iterations inside the call to AlgorithmF5
that is processing $g$.
\item S-pairs with signature $w$ such that $index(w)=index(\Sig(g))=1,\totaldeg(w)=\totaldeg(\Sig(g)),w\prec\Sig(g)$
.They were processed on previous iterations inside the call to Reduce
that is processing $g$.
\end{itemize}
S-pairs from these classes can't be at the middle of processing path
because at the studied state of algorithm the processing is just finished
for $g$ so $P$ and $ToDo$ sets does not contain any non-processed
elements with signatures smaller $\Sig(g)$. The only left thing to
show is proof that processing was started at least one time for all
for S-pairs. This is true for first two classes: the processing of
corresponding S-pairs was started at least one time with the call
to CritPair inside AlgorithmF5 just before the greatest of S-pair
generators was added to $G$. For S-pairs of third class the situation
depend on the total degrees of its generators. If both generators
of S-pair have total degrees $<\totaldeg(g)$ then its processing
is started in CritPair like for the S-pairs from first two classes.
But some S-pairs from the third class can have a signature-greater
generator polynomial $r_{l}$ such that $\totaldeg(r_{l})=\totaldeg(g),\,\Sig(r_{l})\prec\Sig(g)$.
They are processed with the second mentioned processing path so the
processing for such S-pairs is not yet started at the beginning of
last Reduction call. Fortunately, their processing starts inside Reduction
before fixed moment we studying: the procedure selects polynomials
from $ToDo$ in the signature increasing order, so $r_{l}$ is reduced
before $g$ and during $r_{l}$ reduction just before putting $r_{l}$
to $Done$ a call to IsReducible starts processing for all such S-pairs.
\end{proof}
The ideas of \emph{satisfying F5 criterion }and \emph{satisfying Rewritten
criterion} can be extended to arbitrary monomial-multiplied labeled
polynomial $sh,\, h\in G_{g}$:
\begin{defn}
The monomial-multiplied labeled polynomial $sr_{i},\, r_{i}\in G_{g}$
is called \emph{satisfying F5 criterion} if $\varphi_{index(r_{i})+1}(s\Sig(r_{i}))\ne s\Sig(r_{i})$,
where $\varphi_{index(r_{i})+1}$ is operator of normal form w.r.t
$G_{index(r_{i})+1}$.
\end{defn}
This definition is equivalent to $sr_{i}$ being non-normalized labeled
polynomial according to definition 2 in part 5 of \cite{F5-Orig}.
\begin{defn}
The monomial-multiplied labeled polynomial $sr_{i},\, r_{i}\in G_{g}$
is called \emph{satisfying Rewritten criterion} if $\exists j>i$
such that $\Sig(r_{j})|s\Sig(r_{i})$.
\end{defn}
For the case $sr_{i}$ is the S-pair part these definitions are equivalent
to the checks in the algorithm in a sense that S-pair part is rejected
by the algorithm if and only if it satisfies the definition as monomial-multiplied
labeled polynomial. Note that for both criteria holds important property
that if $sr_{i}$ satisfies a criteria then a further multiplied $s_{1}sr_{i}$
$ $satisfies it too.


\section{Representations}


\subsection{Definition}

The idea of representations comes from proof of Theorem 21 in \cite{F5C}.
Representations are used to describe all possible ways how a labeled
polynomial $p$ can be written as an element of $\left(G_{g}\right)$
ideal. The single representation corresponds to writing a labeled
polynomial $p$ as any finite sum of the form
\begin{equation}
p=\sum_{k}m_{k}\cdot b_{i_{k}},\; b_{i_{k}}\in G_{g}\label{eq:Gg-repr-def}
\end{equation}
with coefficients $m_{k}=c_{k}t_{k}\in\mathcal{K}\times T$. 
\begin{defn}
Sum of the form \ref{eq:Gg-repr-def} with all pairs $\left(t_{k},b_{i_{k}}\right)$
distinct is called \emph{$G_{g}$-representation} of $p$. The symbolic
products $m_{k}\cdot b_{i_{k}}$ are called the \emph{elements} of
representation. If we treat this symbolic product as multiplication
we get an labeled polynomial $m_{k}b_{i_{k}}$ corresponding to the
representation element. So $p$ is equal to sum of labeled polynomials,
corresponding to elements of its representation. Also the term \emph{element
signature} will be used for signature of labeled polynomials corresponding
to the element. Two representations are equal if the sets of their
elements are equal.
\end{defn}
Most representations we are interested in have the following additional
property limiting elements signature:
\begin{defn}
The $G_{g}$-representation of $p$ is called \emph{signature-safe}
if $\forall k\,\Sig(m_{k}b_{i_{k}})\preccurlyeq\Sig(p)$.
\end{defn}

\subsection{Examples}
\begin{example}
The first important example of a $G_{g}$-representation is trivial:
the labeled polynomial from $G_{g}$ is equal to sum of one element,
identity-multiplied itself: 
\[
b_{j}=1\cdot b_{j}.
\]

\end{example}
This $G_{g}$-representation is signature-safe. The prohibition of
two elements which have same monomial $t_{k}$ and polynomial $b_{i_{k}}$
ensures that all elements of representation that differ only in field
coefficient $c_{k}$ are combined together by summing field coefficients.
So expressions like $b_{j}=-1\cdot b_{j}+2\cdot b_{j}$ and $b_{j}=2x\cdot b_{k}+1\cdot b_{j}-2x\cdot b_{k}$
are not valid $G_{g}$-representations.
\begin{example}
A labeled polynomial $b_{j}\in G_{g}$ multiplied by arbitrary polynomial
$h$ also have a simple $G_{g}$-representation arising from splitting
$h$ into terms: $h=\sum_{k}m_{k},\, m_{k}\in\mathcal{K}\times T$.
This $G_{g}$-representation has form 
\begin{equation}
b_{j}h=\sum_{k}m_{k}\cdot b_{j}\label{eq:repr-ex-2}
\end{equation}
and is signature-safe too.
\end{example}
A labeled polynomial can have arbitrary number of representations:
for example we can add elements corresponding to a syzygy to any representation
and combine elements with identical monomials and polynomials to get
the correct representation. The result will be representation of the
same polynomial because sum of syzygy elements is equal to 0.
\begin{example}
The product of two polynomial from $G_{g}$ has two representations
of the form (\ref{eq:repr-ex-2}) which differs in syzygy addition:
\end{example}
\[
b_{j}b_{i}=\sum_{k}m_{i_{k}}\cdot b_{j}=\sum_{k}m_{i_{k}}\cdot b_{j}+0=\sum_{k}m_{i_{k}}\cdot b_{j}+\left(\sum_{k}m_{j_{k}}\cdot b_{i}-\sum_{k}m_{i_{k}}\cdot b_{j}\right)=\sum_{k}m_{j_{k}}\cdot b_{i},
\]
where $m_{i_{k}}$ are terms of $b_{i}$ and $m_{j_{k}}$ are terms
of $b_{j}$.
\begin{example}
The zero polynomial has an empty representation and an representation
for every syzygy:
\end{example}
\[
0=\sum_{\emptyset}\mbox{(empty sum)}=\sum_{k}m_{j_{k}}\cdot b_{i}+\sum_{k}(-m_{i_{k}})\cdot b_{j},
\]
where $m_{i_{k}}$ and $m_{j_{k}}$ are same as above.

Another important example of $G_{g}$-representation comes from ideal
and signature definitions. All labeled polynomials computed by the
algorithm are elements of ideal $\left(f_{1},\ldots,f_{m}\right)$.
So any labeled polynomial $p$ can be written as $ $$\sum_{i}f_{i}g_{i}$,
where $g_{i}$ are homogeneous polynomials. All input polynomials
$f_{i}$ belong to $G_{g}$, so $f_{i}g_{i}$ has $G_{g}$-representations
of the form (\ref{eq:repr-ex-2}).
\begin{example}
Those representations sum give the following signature-safe representation:
\[
p=\sum_{k}m_{k}\cdot b_{i_{k}},\, m_{k}\in\mathcal{K}\times T,\, b_{i_{k}}\in\left\{ f_{1},\ldots,f_{m}\right\} \subset G_{g}.
\]
\end{example}
\begin{defn}
This particular case of $G_{g}$-representation where $b_{i_{k}}$
are limited to input polynomials will be called \emph{input-representation}.
\end{defn}
Input representations always has the only element with maximal signature.
This prperty is special to input-representations because generic $G_{g}$-representations
can have multiple elements with same maximal signature -- it is possible
to have $m_{1}\Sig(b_{i_{1}})=m_{2}\Sig(b_{i_{2}})$ while $i_{1}\ne i_{2}$.

The following claim makes important connection between signatures
and input-representations:
\begin{claim}
An admissible labeled polynomial $p$ with known signature $\Sig(p)$
has an input-representation consisting of an element $c\Sig(p)\cdot f_{index(p)}$
and some other elements with smaller signatures.\end{claim}
\begin{proof}
The claimed fact follows from the admissible polynomial definition
in \cite{F5-Orig} referring to function $v$ which correspond to
summing representation elements.
\end{proof}
The theorem 1 of \cite{F5-Orig} states that all polynomials in the
algorithm are admissible, do the above claim will be applied to all
appeared polynomials.
\begin{example}
\label{example-of-having-gg-repr}The last example comes from S-pairs
with a computed $G_{g}$-representation. S-polynomial of $b_{l^{*}}$
and $b_{l_{*}}$ from $G_{g}$ is $p=\overline{u_{l}}poly(b_{l^{*}})-\underline{u_{l}}poly(b_{l_{*}})$.
It is known from reduction process that for such S-pairs $p$ is signature-safe
reduced and the result is added to $G_{g}$ as some labeled polynomial
$b_{l}$. So the $G_{g}$-representation is:
\end{example}
\[
p=\sum_{k}m_{k}\cdot b_{n_{k}}+1\cdot b_{l},
\]
where signatures of $m_{k}\cdot b_{n_{k}}$ elements are smaller than
$\Sig(b_{l})=\Sig(p)$. The value of $l$ is position of $b_{l}$
in ordered list $G_{g}$. In this representation $l$ is greater than
$l^{*}$ and $l_{*}$ because corresponding labeled polynomial $b_{l}$
is added to $R$ at the moment of S-polynomial computation in Spol
or TopReduction so the polynomials $b_{l^{*}}$ and $b_{l_{*}}$ used
to create S-pair already present in $R$ at that moment and the order
of $G_{g}$ correspond to order of $R$.


\subsection{Ordering representations}
\begin{defn}
To order $G_{g}$-representations we start from \emph{representation
elements ordering} $\gtrdot_{1}$: $c_{i}t_{i}\cdot b_{i}\gtrdot_{1}c_{j}t_{j}\cdot b_{j}$
iff $t_{i}\Sig(b_{i})\succ t_{j}\Sig(b_{j})$ or $t_{i}\Sig(b_{i})=t_{j}\Sig(b_{j})\mbox{ and }i<j$
(note the opposite order).
\end{defn}
This ordering is based only on comparison of signatures and positions
of labeled polynomials in the ordered list $G_{g}$ but does not depend
on the field coefficient. The only case in which two elements can't
be ordered is equality of both signatures $t_{i}\Sig(b_{i})=t_{j}\Sig(b_{j})$
and positions in list $i=j$. Position equality means $b_{i}=b_{j}$
which in conjunction with signature equality gives $t_{i}=t_{j}$.
So any two elements that belong to a single $G_{g}$-representation
are comparable with $\lessdot_{1}$ order because they have distinct
$\left(t_{k},b_{k}\right)$ by definition. Below are given some examples
of $\lessdot_{1}$ element ordering for the 3-element list $G_{g}=\left\{ b_{1},\, b_{2},\, b_{3}\right\} $
with ordering $x\mathbf{F}_{i}\succ y\mathbf{F}_{i}$ and signatures
$\Sig(b_{1})=\mathbf{F}_{1},\Sig(b_{2})=\mathbf{F}_{2},\Sig(b_{3})=x\mathbf{F}_{1}$.
\begin{itemize}
\item $y\cdot b_{1}\gtrdot_{1}100y\cdot b_{2}$ because signature of left
side is $\succ$
\item $x\cdot b_{1}\gtrdot_{1}y\cdot b_{1}$ because signature of left side
is $\succ$
\item $-x\cdot b_{1}$ and $2x\cdot b_{1}$ are not comparable because signatures
and list indices are equal
\item $y^{2}\cdot b_{1}\lessdot_{1}y\cdot b_{3}$ because signature of left
side is $\prec$
\item $x^{2}\cdot b_{1}\gtrdot_{1}x\cdot b_{3}$ because signatures are
equal and the list position of left side's labeled polynomial is 1
which is smaller than right side's position 3.
\end{itemize}
To extend this order to entire $G_{g}$-representations consider \emph{ordered
form} of representation consisting of all its elements written in
a list with $\gtrdot_{1}$-decreasing order. This form can be used
for equality testing because if two representations are equal then
they have exactly equal ordered forms.
\begin{defn}
With ordered forms the \emph{$G_{g}$-representations ordering} can
be introduced: the representation $\sum_{k}m'_{k}\cdot b_{i'_{k}}$
is $\lessdot$-smaller than $\sum_{k}m_{k}\cdot b_{i_{k}}$ iff the
ordered form of the first representation is smaller than second's
according to lexicographical extension of $\lessdot_{1}$ ordering
on elements. For the corner case of the one ordered form being beginning
of the other the shorter form is $\lessdot$-smaller. If the greatest
different elements of ordered forms differ only in field coefficient
the representations are not comparable.
\end{defn}
Some examples of this ordering are given for the same as above 3-element
$G_{g}$ list. Note that all $G_{g}$-representations are already
written in ordered forms:
\begin{itemize}
\item $x^{2}\cdot b_{1}+xy\cdot b_{1}+y^{2}b_{1}\gtrdot x^{2}\cdot b_{1}+100y^{2}\cdot b_{1}$
because $xy\cdot b_{1}\gtrdot y^{2}\cdot b_{1}$
\item $x^{2}\cdot b_{1}+100y^{2}\cdot b_{1}\gtrdot x^{2}\cdot b_{1}$ because
the right ordered form is beginning of the left
\item $x^{2}\cdot b_{1}\gtrdot xy\cdot b_{1}+y^{2}\cdot b_{1}+x^{2}\cdot b_{2}$
because $x^{2}\cdot b_{1}\gtrdot xy\cdot b_{1}$
\item $xy\cdot b_{1}+y^{2}\cdot b_{1}+x^{2}\cdot b_{2}\gtrdot y\cdot b_{3}+y^{2}\cdot b_{1}+x^{2}\cdot b_{2}$
because $xy\cdot b_{1}\gtrdot y\cdot b_{3}$
\item $y\cdot b_{3}+y^{2}\cdot b_{1}+x^{2}\cdot b_{2}$ and $2y\cdot b_{3}+y^{2}\cdot b_{2}$
are not comparable because the greatest different elements are $y\cdot b_{3}$
and $2y\cdot b_{3}$.
\end{itemize}
The ordering is compatible with signature-safety:
\begin{thm}
If two representations of $p$ has a relation $\sum_{k}m'_{k}\cdot b_{i'_{k}}\lessdot\sum_{k}m_{k}\cdot b_{i_{k}}$
and the second one is signature-safe representation then the first
one is signature-safe too.\end{thm}
\begin{proof}
This theorem quickly follows from a fact that elements of a $\lessdot$-smaller
representation can't has signatures $\succ$-greater than signatures
of $\gtrdot$-greater representation.
\end{proof}
The key fact allowing to take $\lessdot$-minimal element is well-orderness:
\begin{thm}
The representations are well-ordered with $\lessdot$ ordering.\end{thm}
\begin{proof}
The number of different labeled polynomial positions is finite because
it is equal to $|G_{g}|$ which is finite for fixed $g$. So the existence
of infinite $\gtrdot_{1}$-descending sequence of representation elements
would lead to existence of infinite $\succ$-descending sequence of
signatures. Combining this with well-orderness of signatures with
ordering $\prec$ we get the proof for well-orderness of elements
with ordering $\lessdot{}_{1}$.

The straightforward proof for $\lessdot$-well-orderness of representations
following from $\lessdot_{1}$-well-orderness of elements is not very
complex but to skip its strict details the theorem 2.5.5 of \cite{Multiset-Order}
will be referenced. It states well-orderness of finite multiset with
an lexicographically extended ordering of well-ordered elements. This
applies to the representations because they form a subset in the finite
multiset of representation elements.
\end{proof}

\subsection{Sequence of representations}

The idea of this part is constructing a finite sequence of strictly
$\lessdot$-descending signature-safe $G_{g}$-representations for
a given labeled polynomial $mh,\, m\in\mathcal{K}\times T,\, h\in G_{g}$
with $\Sig(mh)\prec\Sig(g)$. The first signature-safe representation
in the sequence is $mh=m\cdot h$, the last representation is $mh=\sum_{k}m_{k}\cdot b_{i_{k}}$
with elements having the following properties $\forall k$:
\begin{enumerate}
\item $m_{k}b_{i_{k}}$ does not satisfy F5 criterion.
\item $m_{k}b_{i_{k}}$ does not satisfy Rewritten criterion.
\item $\HM(m_{k}b_{i_{k}})\leqslant\HM(mh)$
\end{enumerate}
The proof of such sequence existence is very similar to Theorem 21
of \cite{F5C} and is based on a fact, that if a some signature-safe
representation of $mh$ contains an element $m_{K}\cdot b_{i_{K}}$
not having one of the properties then a $\lessdot$-smaller representation
can be constructed. Though the exact construction differ for three
cases but the replacement scheme is the same:
\begin{itemize}
\item a some element $m_{K'}\cdot b_{i_{K'}}$ in $mh$ representation is
selected. Note that $K'$ in some cases is not equal to $K$
\item some representation $m_{K'}b_{i_{K'}}=\sum_{l}m_{l}\cdot b_{i_{l}}$
is constructed for this element.
\item it is shown that constructed representation is $\lessdot$-smaller
than representation $m_{K'}b_{i_{K'}}=m_{K'}\cdot b_{i_{K'}}$ 
\end{itemize}
Construction of such representation for $m_{K'}\cdot b_{i_{K'}}$
allows application of the following lemma:
\begin{lem}
If an element $m_{K'}\cdot b_{i_{K'}}$ of signature-safe representation
$mh=\sum_{k}m_{k}\cdot b_{i_{k}}$ has an representation $m_{K'}b_{i_{K'}}=\sum_{l}m_{l}\cdot b_{i_{l}}$
which is $\lessdot$-smaller than representation $m_{K'}b_{i_{K'}}=m_{K'}\cdot b_{i_{K'}}$
then $mh$ has a signature-safe representation $\lessdot$-smaller
than $mh=\sum_{k}m_{k}\cdot b_{i_{k}}$.\end{lem}
\begin{proof}
We replace $m_{K'}\cdot b_{i_{K'}}$ in $mh=\sum_{k}m_{k}\cdot b_{i_{k}}$
by $\sum_{l}m_{l}\cdot b_{i_{l}}$ and combine coefficients near elements
with both monomial and polynomial equal, so a modified representation
for $mh$ appears. Is is $\lessdot$-smaller than $mh=\sum_{k}m_{k}\cdot b_{i_{k}}$
because all elements $\gtrdot_{1}$-greater than $m_{K'}\cdot b_{i_{K'}}$
are identical in both representations if they present but the element
 $m_{K'}\cdot b_{i_{K'}}$ is contained in original representation
but not in the modified. And all other elements in representations
are $\lessdot_{1}$-smaller than $m_{K'}\cdot b_{i_{K'}}$ so they
doesn't influence the comparison. The comparison holds even in a corner
case when all elements are discarded while combining coefficients.
This case can appear if the original representation is equal to $mh=m_{K'}\cdot b_{i_{K'}}+\sum_{l}(-m_{l})\cdot b_{i_{l}}$
what leads to modified representation $mh=0$ with zero elements which
is $\lessdot$-smaller than any non-empty representation.
\end{proof}
Now it will be shown that replacement scheme can be performed if the
representation contains an element not satisfying at least one of
three properties.
\begin{lem}
If a signature-safe $G_{g}$-representation $mh=\sum_{k}m_{k}\cdot b_{i_{k}}$
does not satisfy property 1 then there exists an element $m_{K'}\cdot b_{i_{K'}}$
having $G_{g}$-representation $m_{K'}b_{i_{K'}}=\sum_{l}m_{l}\cdot b_{i_{l}}$
which is $\lessdot$-smaller than representation $m_{K'}b_{i_{K'}}=m_{K'}\cdot b_{i_{K'}}$
.
\end{lem}
An element not having the first property does satisfy the F5 criterion
and the idea is to use that $m_{K}\Sig(b_{i_{K}})$ is not the minimal
signature of $m_{K}b_{i_{K}}$ like in Theorem 20 of \cite{F5C}.
$K'=K$ is taken for this case.
\begin{proof}
Consider input-representation of $m_{K}b_{i_{K}}$ with signature
of $\gtrdot_{1}$-maximal element equal to $m_{K}\Sig(b_{i_{K}})=s_{0}\mathbf{F}_{j_{0}}$:

\begin{equation}
m_{K}b_{i_{K}}=c_{0}s_{0}\cdot f_{j_{0}}+\sum_{l}m_{l}\cdot f_{i_{l}}.\label{eq:input-repr-case1}
\end{equation}
From the satisfying F5 criterion $s_{0}$ can be expressed like $s_{0}=s_{1}\HM(f_{j_{1}}),\, j_{1}>j_{0}$
so $s_{0}f_{j_{0}}=s_{1}f_{j_{0}}f_{j_{1}}-s_{1}(f_{j_{1}}-\HM(f_{j_{1}}))f_{j_{0}}$.
From this we can write another representation for $m_{K}b_{i_{K}}$,
assuming $m_{0i}$ are sorted terms of $f_{j_{0}}$, $m_{1i}$ are
sorted terms of $f_{j_{1}}$ and $N_{0,}N_{1}$ are number of terms
in those polynomials:

\[
m_{K}b_{i_{K}}=\sum_{i=1}^{N_{0}}c_{0}s_{1}m_{0i}\cdot f_{j_{1}}+\sum_{i=2}^{N_{1}}-c_{0}s_{1}m_{1i}\cdot f_{j_{0}}+\sum_{l}m_{l}\cdot f_{i_{l}}.
\]
This representation is $\lessdot$-smaller than $m_{K}\cdot b_{i_{K}}$
because signatures of all elements are smaller than $s_{0}\mathbf{F}_{j_{0}}$.
For the elements of the third sum $\sum_{l}m_{l}\cdot f_{i_{l}}$
this follows from \textbf{\ref{eq:input-repr-case1}}, where those
elements are smaller elements of input-representation. For the elements
of the first sum $\sum_{i=1}^{N_{0}}c_{0}s_{1}m_{0i}\cdot f_{j_{1}}$
this follows from the position inequality $j_{1}>j_{0}$. And for
the second sum we use the equality in term and signature orderings:
all terms $m_{1i},\, i\geqslant2$ are smaller than $m_{11}$, so
the signatures are: $s_{1}m_{1i}\mathbf{F}_{j_{0}}\prec s_{1}m_{11}\mathbf{F}_{j_{0}}=s_{0}\mathbf{F}_{j_{0}}$. \end{proof}
\begin{lem}
If a signature-safe $G_{g}$-representation $mh=\sum_{k}m_{k}\cdot b_{i_{k}}$
with $\Sig(mh)\prec\Sig(g)$ does not satisfy property 2 then there
exists an element $m_{K'}\cdot b_{i_{K'}}$ having $G_{g}$-representation
$m_{K'}b_{i_{K'}}=\sum_{l}m_{l}\cdot b_{i_{l}}$ which is $\lessdot$-smaller
than representation $m_{K'}b_{i_{K'}}=m_{K'}\cdot b_{i_{K'}}$ \textup{.}
\end{lem}
For the elements not satisfying case 2 the $\lessdot$-smaller representation
is created in a way used in Proposition 17 of \cite{F5C}. $K'=K$
is taken for this case too.
\begin{proof}
Assume that $\Sig(m_{K}b_{i_{K}})=s_{0}\mathbf{F}_{j_{0}}$ and it
is rewritten by labeled polynomial $b_{i'}$ from $R$. Because the
representation is signature-safe we have $\Sig(b_{i'})\preccurlyeq s_{0}\mathbf{F}_{j_{0}}\preccurlyeq\Sig(mh)\prec\Sig(g)$.
So $b_{i'}$ was processed in TopReduction before $g$. Since $b_{i'}$
is rewriter we have $b_{i'}\ne0$. All this gives the fact that $b_{i'}$
does present not only in $R$ but in $G_{g}$ too so it can be used
as a polynomial of $G_{g}$-representation element. From the Rewritten
criterion definition we know that $i'>i_{K}$ and the existence of
$s'\in T$ such that $s'\Sig(b_{i'})=s_{0}\mathbf{F}_{j_{0}}$. So,
for the $m_{K}b_{i_{K}}$ there is an input-representation \ref{eq:input-repr-case1}
and for the $s'b_{i'}$ the input-representation is:

\[
s'b_{i'}=c's_{0}\cdot f_{j_{0}}+\sum_{l'}m_{l'}\cdot f_{i_{l'}}.
\]
A $G_{g}$-representation for $c_{0}s_{0}f_{j_{0}}$ can be acquired
with transformation of the above expression:

\[
c_{0}s_{0}f_{j_{0}}=c'^{-1}c_{0}s'\cdot b_{i'}+\sum_{l'}-c'^{-1}c_{0}m_{l'}\cdot f_{i_{l'}}.
\]
Using this to replace the first element in \ref{eq:input-repr-case1}
we get the wanted result:
\[
m_{K}b_{i_{K}}=c'^{-1}c_{0}s'\cdot b_{i'}+\sum_{l'}-c'^{-1}c_{0}m_{l'}\cdot f_{i_{l'}}+\sum_{l}m_{l}\cdot f_{i_{l}}
\]
It is $\lessdot$-smaller than $m_{K}b_{i_{K}}=m_{K}\cdot b_{i_{K}}$
because elements of both sums has signatures smaller than $s_{0}\mathbf{F}_{j_{0}}$,
and for the first element $ $$\Sig(c'^{-1}c_{0}s'\cdot b_{i'})=\Sig(m_{K}\cdot b_{i_{K}})=s_{0}\mathbf{F}_{j_{0}}$
but $i'>i_{K}$, so applying the $\lessdot_{1}$-comparison rule for
equal signatures and different list positions we get that element
$c'^{-1}c_{0}s'\cdot b_{i'}$ is $\lessdot_{1}$-smaller than $m_{K}\cdot b_{i_{K}}$
too.\end{proof}
\begin{lem}
If a signature-safe representation $mh=\sum_{k}m_{k}\cdot b_{i_{k}}$
with $\Sig(mh)\prec\Sig(g)$ satisfies properties 1 and 2 but does
not satisfy property 3 then there exists an element $m_{K'}\cdot b_{i_{K'}}$
having representation $m_{K'}b_{i_{K'}}=\sum_{l}m_{l}\cdot b_{i_{l}}$
which is $\lessdot$-smaller than representation $m_{K'}b_{i_{K'}}=m_{K'}\cdot b_{i_{K'}}$.\end{lem}
\begin{proof}
There exists at least one element $m_{K}\cdot b_{i_{K}}$ that does
not satisfy property 3. Let $m_{\max}$ be the maximal $\HM$ of labeled
polynomials corresponding to representation elements and $H_{\max}$
be a list of elements where $m_{\max}$ is achieved. Select $K'$
to be the index of the $\gtrdot_{1}$-greatest representation element
in $H_{\max}$. We have $ $$\HM(m_{K'}b_{i_{K'}})=m_{\max}\geqslant\HM(m_{K}b_{i_{K}})>\HM(mh)$,
so the HM of sum of all elements except $K'$ is equal to $\HM(mh-m_{K'}b_{i_{K'}})=\HM(m_{K'}b_{i_{K'}})=m_{\max}$,
so there is another element $K''$ having $\HM(m_{K''}b_{i_{K''}})=m_{\max}$.
So, $m_{K''}\cdot b_{i_{K''}}\in H_{\max}$ and $m_{K''}\cdot b_{i_{K''}}\lessdot_{1}m_{K'}\cdot b_{i_{K'}}$
because of $ $$m_{K'}\cdot b_{i_{K'}}$ $\gtrdot_{1}$-maximality
in $H_{\max}$.

The $\HM(m_{K''}b_{i_{K''}})=\HM(m_{K'}b_{i_{K'}})$ means that a
critical pair of $b_{i_{K'}}$ and $b_{i_{K''}}$ has the form $[m'^{-1}m_{\max},\, m'^{-1}m_{K'},\, b_{i_{K'}},\, m'^{-1}m_{K''},\, b_{i_{K''}}]$
where $m'=\mbox{gcd}(m_{K'},m_{K''})$. Let $q$ be corresponding
S-polynomial. Then $m'\Sig(q)\preccurlyeq\Sig(mh)\prec\Sig(g)$ because
$m'\Sig(q)=\Sig(m_{K'}b_{i_{K'}})$ and the representation is signature-safe.
The S-polynomial parts $m'^{-1}m_{K'}b_{i_{K'}}$ and $m'^{-1}m_{K''}b_{i_{K''}}$
doesn't satisfy F5 and Rewritten criteria because their forms multiplied
by $m'$ are $m_{K'}b_{i_{K'}}$ and $m_{K''}b_{i_{K''}}$ -- labeled
polynomials corresponding to elements which are known not to satisfy
both criteria by assumption. Therefore $m'\Sig(q)\prec\Sig(g)$ and
$\Sig(q)\prec\Sig(g)$. It follows from this with theorem \ref{thm:Exist-gg-repr}
that the S-pair $(b_{i_{K'}},b_{i_{K''}})$ is S-pair with computed
$G_{g}$-representation, what means that there is an representation
described in example \ref{example-of-having-gg-repr} :

\[
q=1\cdot b_{i'}+\sum_{l}m_{l}\cdot b_{i_{l}},
\]
satisfying the properties shown after that example: $\Sig(q)=\Sig(b_{i'})$,
$\forall l\,\Sig(q)\succ\Sig(m_{l}b_{i_{l}})$ and $i'>K'$.

From the other hand we have $m'q=c_{0}m_{K'}b_{i_{K'}}-c_{1}m_{K''}b_{i_{K''}}$,
so we get the following representation:
\[
m_{K'}b_{i_{K'}}=c_{0}^{-1}c_{1}m_{K''}\cdot b_{i_{K''}}+c_{0}^{-1}m'\cdot b_{i'}+\sum_{l}c_{0}^{-1}m'm_{l}\cdot b_{i_{l}}.
\]
It is $\lessdot$-smaller than $m_{K'}b_{i_{K'}}=m_{K'}\cdot b_{i_{K'}}$: 

$m_{K''}\cdot b_{i_{K''}}$ was already compared to $m_{K'}\cdot b_{i_{K'}}$ 

$m'\cdot b_{i'}$ has the same signature but greater position $i'>i_{K'}$

the last sum contains elements with signatures smaller than $m'\Sig(b_{i'})=\Sig(m_{K'}\cdot b_{i_{K'}})$.\end{proof}
\begin{thm}
\label{thm:exist-smaller-signature-safe-representation}A signature-safe
representation $mh=\sum_{k}m_{k}\cdot b_{i_{k}}$ with $\Sig(mh)\prec\Sig(g)$
either satisfies properties 1-3 or there exists a signature-safe representation
$mh=\sum_{l}m_{l}\cdot b_{i_{l}}$ which is $\lessdot$-smaller than
\textup{$\sum_{k}m_{k}\cdot b_{i_{k}}$.}\end{thm}
\begin{proof}
This theorem quickly follows from four previous lemmas together
\end{proof}
This leads to main result:
\begin{thm}
For any labeled polynomial $mh,\, m\in\mathcal{K}\times T,\, h\in G_{g}$
with $\Sig(mh)\prec\Sig(g)$ there exists a signature-safe $G_{g}$-representation
$mh=\sum_{k}m_{k}\cdot b_{i_{k}}$ that satisfies properties 1-3\textup{.}\end{thm}
\begin{proof}
Start with representation $mh=m\cdot h$ and begin replacing it by
$\lessdot$-smaller representation from theorem \ref{thm:exist-smaller-signature-safe-representation}
until the representation satisfying properties 1-3 appears. The finiteness
of the process is guaranteed by $\lessdot$-well-orderness.
\end{proof}
This result may be interesting by itself, but for the purposes of
proving termination only one corollary is needed:
\begin{cor}
\label{cor:all-needed-for-terminaton}Consider an arbitrary polynomial
$f$ without any restrictions on its signature. If there exists a
signature-safe reductor $f'\in G_{g}$ for $f$ with $\Sig(f')\frac{\HM(f)}{\HM(f')}\prec\Sig(g)$
then $G_{g}$ contains a signature-safe reductor for $f$ that is
not rejected by F5 and Rewritten criteria.\end{cor}
\begin{proof}
Let $mf',\, m=\frac{\HM(f)}{\HM(f')}\in\mathcal{K}\times T,\, f'\in G_{g}$
be a multiplied reductor with $\Sig(mf')\prec\Sig(g)$. From the previous
theorem we can find representation $mf'=\sum_{k}m_{k}\cdot b_{i_{k}}$
that satisfies properties 1-3. Property 3 means that there is no elements
with HMs greater than $mf'$ so because sum of all elements has HM
equal to $\HM(mf')$ there exists an element $K$ that achieves HM
equality: $\HM(m_{K}\cdot b_{i_{K}})=\HM(mf')=\HM(f_{1}')$. Since
the representation is signature-safe $\Sig(m_{K}\cdot b_{i_{K}})\preccurlyeq\Sig(mf')\prec\Sig(f)$
so $m_{K}b_{i_{K}}$ is a signature-safe reductor for $f$ and properties
1-2 ensure that $m_{K}b_{i_{K}}$ does not satisfy criteria.
\end{proof}

\section{Finding contradiction with the criteria enabled}

Now return to the result of theorem \ref{thm:f_g_3_props} which states
for the case of algorithm non-termination existence of a polynomials
$f',f\in G$ such that $\HM(f')|\HM(f)$, $\frac{\HM(f')}{\Sig(f')}>\frac{\HM(f)}{\Sig(f)}$.
Using this result and last corollary we construct two polynomials
leading to contradiction for the case of algorithm non-termination.
\begin{thm}
\label{thm:always-exist-ok-reductor}If the algorithm does not terminate
for some input then after some finite step the set \textup{$G\cup Done$}
contains a pair of labeled polynomials $f'_{1},f$ where:
\begin{itemize}
\item $f'_{1}$ is added to \textup{$G\cup Done$} before $f$
\item $t_{1}f'_{1}$ does not satisfy F5 and Rewritten criteria, where $t_{1}=\frac{\HM(f)}{\HM(f'_{1})}$
\item $f'_{1}$ is signature-safe reductor for $f$.
\end{itemize}
\end{thm}
\begin{proof}
Let $f',f$ be polynomials from the theorem \ref{thm:f_g_3_props}
an define $t=\frac{\HM(f)}{\HM(f')}$. We have $f\in G$ so the above
theory about representations can be applied to the fixed value of
$g$ equal to $f$ and we can speak about $G_{f}$ set and $G_{f}$-representations.
Because $tf'$ is a signature-safe reductor for $f$ we have $\Sig(f')t\prec\Sig(f)$
and the corollary \ref{cor:all-needed-for-terminaton} can be applied
to find a signature-safe reductor $t_{1}f'_{1}$ for $f$ which does
not satisfy criteria. Also it is known to belong to $G_{f}$, so during
the algorithm execution $f'_{1}$ was appended to $G\cup Done$ before
$f$.\end{proof}
\begin{thm}
The original $ $F$_{5}$ algorithm as described in \cite{F5-Orig}
does terminate for any input.\end{thm}
\begin{proof}
We are going o show that the existence of polynomials $f'_{1},f$
from the theorem \ref{thm:always-exist-ok-reductor} leads to contradiction.
Consider the call to TopReduction after which the polynomial $f$
was inserted in $Done$. That call returns polynomial $f$ as first
part of TopReduction return value, so the value returned by IsReducible
is empty set. It means that one of conditions (a) - (d) was not satisfied
for all polynomials in $G\cup Done$ including $f'_{1}$. This is
not possible because:
\begin{itemize}
\item (a) is satisfied because $f'_{1}$ is a reductor for $f$ from the
theorem \ref{thm:always-exist-ok-reductor} 
\item (b) and (c) are satisfied because $\frac{\HM(f)}{\HM(f'_{1})}f'_{1}$
does not satisfy F5 and Rewritten criteria from the theorem \ref{thm:always-exist-ok-reductor} 
\item (d) is satisfied because $f'_{1}$ is a signature-safe reductor for
$f$ from the theorem \ref{thm:always-exist-ok-reductor}.
\end{itemize}
\end{proof}

\section{Conclusions}

This paper shows that original F$_{5}$ algorithm terminates for any
homogeneous input. However, it does not give any limit on number of
operations. The simplest proof of the termination of Buchberger algorithm
is based on Noetherian property and doesn't give any such limit too.
Unfortunately the termination proof given here is quite different
in structure compared to the proof of Buchberger algorithm termination,
so this proof doesn't show that F$_{5}$ is more efficient than Buchberger
in any sense. Unlike this the termination of the modified versions
of F$_{5}$ algorithm in \cite{term-mod-1,term-mod-2,Modifying-for-termination}
is shown in a way analogous to Buchberger algorithm and there is room
for comparison of their efficiency with Buchberger's one.

From the point of view of practical computer algebra computations
there is a question about efficiency of the modified versions compared
to original F$_{5}$. Unfortunately the only strict fact in this area
that follows from this paper is that both modified and original versions
terminate. The modified versions can spend more time in additional
termination checks. But for some cases it is possible that those checks
can allow the termination of modified versions before original so
the modified version performs smaller number of reductions. So it
is possible that for some inputs the original algorithm is faster
and for others the modified version. Some experimental timings in
Table 1 in \cite{Modifying-for-termination} shows that both cases
are possible in practice but the difference in time is insignificant.
So the question about efficiency of original F$_{5}$ compared to
modified versions is open.

This proof uses three properties of original F$_{5}$ that are absent
or optional in some F$_{5}$-like algorithms: the homogeneity of input
polynomials, the presence of Rewritten criterion and the equality
of monomial order $<$ and signature order $\prec$. The possibility
of extending the termination proof to the modified algorithms without
these properties is open question. There is an unproved idea that
the proof can be modified to remove reliance on the first two properties
but not on the third property of orders equality because it is key
point of coming to a contradiction form the result of theorem \ref{thm:f_g_3_props}.\\



\thanks{The author would like to thank Christian Eder, Jean-Charles Faug�re,
Amir Hashemi, John Perry, Till Stagers and Alexey Zobnin for inspiring
me on investigations in this area by their papers and comments. Thanks!}
\begin{thebibliography}{References}
\bibitem{Arri-Perry-Revised}Alberto Arri and John Perry. The F5 criterion
revised

\bibitem{key-2}Alexey Zobnin. Generalization of the F5 algorithm
for calculating Gr�bner bases for polynomial ideals

\bibitem{F5C}Christian Eder, John Perry. F5C: a variant of Faug�re's
F5 algorithm with reduced Gr�bner bases

\bibitem{Modifying-for-termination}Christian Eder, Justin Gash, and
John Perry. Modifying Faug�re\textquoteright{}s F5 algorithm to ensure
termination

\bibitem{Multiset-Order}Franz Baader and Tobias Nipkow. Term Rewriting
and All That

\bibitem{key-1}Gw�nol� Ars, Amir Hashemi. Extended F 5 criteria.

\bibitem{term-mod-1}Gw�nol� Ars. Applications des bases de Gr�bner
� la cryptographie

\bibitem{F5-Orig}Jean-Charles Faug�re. A new effi{}cient algorithm
for computing Gr�bner bases without reduction to zero (F$_{5}$)

\bibitem{term-mod-2}Justin M. Gash. On efficient computation of Gr�bner
bases

\bibitem{F5-Revisited}Till Stagers. Faug�re\textquoteright{}s F5
Algorithm Revisited\end{thebibliography}

\end{document}
