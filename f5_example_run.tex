Мы будем рассматривать работу алгоритма в кольце многочленов над конечным полем $\mathbb{Z}_{2}$, с мономиальным порядком $\mbox{degrevlex}(x>y>z>t>w)$.\\*
 \textbf{Инициализация:} IncrementalF5 записывает в массив $R$ входные многочлены, добавляя к ним сигнатуры с единичным мономом и индексом, равным порядковому номеру во входных данных:\\* $R\leftarrow[ [1]=\lpoly{\monone}{1}{xy+tw},$\\*
$[2]=\lpoly{\monone}{2}{yz^{2}+ztw},$\\*
$[3]=\lpoly{\monone}{3}{x^{2}+w^{2}}]$\\*
$G_3$ присваивается значение $\{3\}$.\\*
\textbf{Шаг 1:} AlgorithmF5 обрабатывает многочлен $R[2]$ и формирует по предыдущему базису $G_{2}=G_3\cup \{2\}$.\\*
CritPair: S-пара $(x^{2}\cdot R[2],yz^{2}\cdot R[3])$  отбрасывается $\varphi$-проверкой, ибо моном сигнатуры $x^{2}\cdot\Sss{\monone}{2}$ делится на старший моном $x^{2}$ базиса прошлого шага.\\*
Результирующим базисом шага оказываются многочлены R с позициями из $G_2=\{3, 2\}$.\\*
\textbf{Шаг 2:} AlgorithmF5 обрабатывает многочлен $R[1]$ и формирует по предыдущему базису $G_{1}=G_2\cup \{1\}$.\\*
CritPair: S-пара $(x\cdot R[1],y\cdot R[3])$ добавляется в $P$.\\*
CritPair: S-пара $(z^{2}\cdot R[1],x\cdot R[2])$ добавляется в $P$.\\*
AlgorithmF5: выбор S-пар степени 3.\\*
Spol: S-многочлен $x\cdot R[1] - y\cdot R[3]$ добавляется как $R[4]=\lpoly{x}{1}{xtw+yw^{2}}$ и попадает в правила, делая Rule[1] равным $[4]$.\\*
TopReduction: поиск редукторов $R[4]$.\\*
TopReduction: $R[4]$ не может быть далее редуцирован и добавляется в Done.\\*
CritPair: S-пара $(x\cdot R[4],tw\cdot R[3])$  отбрасывается $\varphi$-проверкой, ибо моном сигнатуры $x\cdot\Sss{x}{1}$ делится на старший моном $x^{2}$ базиса прошлого шага.\\*
CritPair: S-пара $(yz^{2}\cdot R[4],xtw\cdot R[2])$  отбрасывается $\varphi$-проверкой, ибо моном сигнатуры $yz^{2}\cdot\Sss{x}{1}$ делится на старший моном $yz^{2}$ базиса прошлого шага.\\*
CritPair: S-пара $(y\cdot R[4],tw\cdot R[1])$ добавляется в $P$.\\*
AlgorithmF5: добавляет в $G_{1}$ следующее множество позиций многочленов в массиве $R: \{4\}$.\\*
AlgorithmF5: выбор S-пар степени 4.\\*
Spol: S-многочлен $z^{2}\cdot R[1] - x\cdot R[2]$ добавляется как $R[5]=\lpoly{z^{2}}{1}{xztw+z^{2}tw}$ и попадает в правила, делая Rule[1] равным $[5, 4]$.\\*
Spol: S-многочлен $y\cdot R[4] - tw\cdot R[1]$ добавляется как $R[6]=\lpoly{xy}{1}{y^{2}w^{2}+t^{2}w^{2}}$ и попадает в правила, делая Rule[1] равным $[6, 5, 4]$.\\*
TopReduction: поиск редукторов $R[5]$.\\*
TopReduction: потенциальный редуктор $R[4]$ не может произвести редукцию с сохранением сигнатуры. S-многочлен $R[5]$ и $R[4]$ добавляется как $R[7]=\lpoly{xz}{1}{z^{2}tw+yzw^{2}}$ и попадает в правила, делая Rule[1] равным $[7, 6, 5, 4]$.\\*
TopReduction: поиск редукторов $R[5]$.\\*
TopReduction: потенциальный редуктор $R[4]$ отбрасывается критерием перезаписи, поскольку сигнатура $z\cdot\Sss{x}{1}$ делится на сигнатуру элемента $R[7]$.\\*
TopReduction: $R[5]$ не может быть далее редуцирован и добавляется в Done.\\*
TopReduction: поиск редукторов $R[7]$.\\*
TopReduction: $R[7]$ не может быть далее редуцирован и добавляется в Done.\\*
TopReduction: поиск редукторов $R[6]$.\\*
TopReduction: $R[6]$ не может быть далее редуцирован и добавляется в Done.\\*
CritPair: S-пара $(x\cdot R[5],ztw\cdot R[3])$ добавляется в $P$.\\*
CritPair: S-пара $(yz\cdot R[5],xtw\cdot R[2])$  отбрасывается $\varphi$-проверкой, ибо моном сигнатуры $yz\cdot\Sss{z^{2}}{1}$ делится на старший моном $yz^{2}$ базиса прошлого шага.\\*
CritPair: S-пара $(y\cdot R[5],ztw\cdot R[1])$  отбрасывается $\varphi$-проверкой, ибо моном сигнатуры $y\cdot\Sss{z^{2}}{1}$ делится на старший моном $yz^{2}$ базиса прошлого шага.\\*
CritPair: S-пара $(\monone\cdot R[5],z\cdot R[4])$ добавляется в $P$.\\*
CritPair: S-пара $(x^{2}\cdot R[7],z^{2}tw\cdot R[3])$  отбрасывается $\varphi$-проверкой, ибо моном сигнатуры $x^{2}\cdot\Sss{xz}{1}$ делится на старший моном $x^{2}$ базиса прошлого шага.\\*
CritPair: S-пара $(y\cdot R[7],tw\cdot R[2])$ добавляется в $P$.\\*
CritPair: S-пара $(xy\cdot R[7],z^{2}tw\cdot R[1])$  отбрасывается $\varphi$-проверкой, ибо моном сигнатуры $xy\cdot\Sss{xz}{1}$ делится на старший моном $x^{2}$ базиса прошлого шага.\\*
CritPair: S-пара $(x\cdot R[7],z^{2}\cdot R[4])$  отбрасывается $\varphi$-проверкой, ибо моном сигнатуры $x\cdot\Sss{xz}{1}$ делится на старший моном $x^{2}$ базиса прошлого шага.\\*
CritPair: S-пара $(x\cdot R[7],z\cdot R[5])$  отбрасывается $\varphi$-проверкой, ибо моном сигнатуры $x\cdot\Sss{xz}{1}$ делится на старший моном $x^{2}$ базиса прошлого шага.\\*
CritPair: S-пара $(x^{2}\cdot R[6],y^{2}w^{2}\cdot R[3])$  отбрасывается $\varphi$-проверкой, ибо моном сигнатуры $x^{2}\cdot\Sss{xy}{1}$ делится на старший моном $x^{2}$ базиса прошлого шага.\\*
CritPair: S-пара $(z^{2}\cdot R[6],yw^{2}\cdot R[2])$  отбрасывается $\varphi$-проверкой, ибо моном сигнатуры $z^{2}\cdot\Sss{xy}{1}$ делится на старший моном $yz^{2}$ базиса прошлого шага.\\*
CritPair: S-пара $(x\cdot R[6],yw^{2}\cdot R[1])$  отбрасывается $\varphi$-проверкой, ибо моном сигнатуры $x\cdot\Sss{xy}{1}$ делится на старший моном $x^{2}$ базиса прошлого шага.\\*
CritPair: S-пара $(xt\cdot R[6],y^{2}w\cdot R[4])$  отбрасывается $\varphi$-проверкой, ибо моном сигнатуры $xt\cdot\Sss{xy}{1}$ делится на старший моном $x^{2}$ базиса прошлого шага.\\*
CritPair: S-пара $(xzt\cdot R[6],y^{2}w\cdot R[5])$  отбрасывается $\varphi$-проверкой, ибо моном сигнатуры $xzt\cdot\Sss{xy}{1}$ делится на старший моном $x^{2}$ базиса прошлого шага.\\*
CritPair: S-пара $(z^{2}t\cdot R[6],y^{2}w\cdot R[7])$  отбрасывается $\varphi$-проверкой, ибо моном сигнатуры $z^{2}t\cdot\Sss{xy}{1}$ делится на старший моном $yz^{2}$ базиса прошлого шага.\\*
AlgorithmF5: добавляет в $G_{1}$ следующее множество позиций многочленов в массиве $R: \{5, 7, 6\}$.\\*
AlgorithmF5: выбор S-пар степени 4.\\*
Spol: S-многочлен $\monone\cdot R[5] - z\cdot R[4]$ отбрасывается критерием перезаписи, поскольку сигнатура $z\cdot\Sss{x}{1}$ делится на сигнатуру элемента $R[7]$.\\*
AlgorithmF5: выбор S-пар степени 5.\\*
Spol: S-многочлен $x\cdot R[5] - ztw\cdot R[3]$ отбрасывается критерием перезаписи, поскольку сигнатура $x\cdot\Sss{z^{2}}{1}$ делится на сигнатуру элемента $R[7]$.\\*
Spol: S-многочлен $y\cdot R[7] - tw\cdot R[2]$ добавляется как $R[8]=\lpoly{xyz}{1}{y^{2}zw^{2}+zt^{2}w^{2}}$ и попадает в правила, делая Rule[1] равным $[8, 7, 6, 5, 4]$.\\*
TopReduction: поиск редукторов $R[8]$.\\*
TopReduction: потенциальный редуктор $R[6]$ отбрасывается критерием перезаписи, поскольку сигнатура $z\cdot\Sss{xy}{1}$ делится на сигнатуру элемента $R[8]$.\\*
TopReduction: $R[8]$ не может быть далее редуцирован и добавляется в Done.\\*
CritPair: S-пара $(x^{2}\cdot R[8],y^{2}zw^{2}\cdot R[3])$  отбрасывается $\varphi$-проверкой, ибо моном сигнатуры $x^{2}\cdot\Sss{xyz}{1}$ делится на старший моном $x^{2}$ базиса прошлого шага.\\*
CritPair: S-пара $(z\cdot R[8],yw^{2}\cdot R[2])$  отбрасывается $\varphi$-проверкой, ибо моном сигнатуры $z\cdot\Sss{xyz}{1}$ делится на старший моном $yz^{2}$ базиса прошлого шага.\\*
CritPair: S-пара $(x\cdot R[8],yzw^{2}\cdot R[1])$  отбрасывается $\varphi$-проверкой, ибо моном сигнатуры $x\cdot\Sss{xyz}{1}$ делится на старший моном $x^{2}$ базиса прошлого шага.\\*
CritPair: S-пара $(xt\cdot R[8],y^{2}zw\cdot R[4])$  отбрасывается $\varphi$-проверкой, ибо моном сигнатуры $xt\cdot\Sss{xyz}{1}$ делится на старший моном $x^{2}$ базиса прошлого шага.\\*
CritPair: S-пара $(xt\cdot R[8],y^{2}w\cdot R[5])$  отбрасывается $\varphi$-проверкой, ибо моном сигнатуры $xt\cdot\Sss{xyz}{1}$ делится на старший моном $x^{2}$ базиса прошлого шага.\\*
CritPair: S-пара $(zt\cdot R[8],y^{2}w\cdot R[7])$  отбрасывается $\varphi$-проверкой, ибо моном сигнатуры $zt\cdot\Sss{xyz}{1}$ делится на старший моном $yz^{2}$ базиса прошлого шага.\\*
CritPair: S-пара $(\monone\cdot R[8],z\cdot R[6])$ добавляется в $P$.\\*
AlgorithmF5: добавляет в $G_{1}$ следующее множество позиций многочленов в массиве $R: \{8\}$.\\*
AlgorithmF5: выбор S-пар степени 5.\\*
Spol: S-многочлен $\monone\cdot R[8] - z\cdot R[6]$ отбрасывается критерием перезаписи, поскольку сигнатура $z\cdot\Sss{xy}{1}$ делится на сигнатуру элемента $R[8]$.\\*
Результирующим базисом шага оказываются многочлены R с позициями из $G_1=\{3, 2, 1, 4, 5, 7, 6, 8\}$.\\*
После убирания сигнатур финальный нередуцированный базис принимает вид:\\*
$R[3]=x^{2}+w^{2}$\\*
$R[2]=yz^{2}+ztw$\\*
$R[1]=xy+tw$\\*
$R[4]=xtw+yw^{2}$\\*
$R[5]=xztw+z^{2}tw$\\*
$R[7]=z^{2}tw+yzw^{2}$\\*
$R[6]=y^{2}w^{2}+t^{2}w^{2}$\\*
$R[8]=y^{2}zw^{2}+zt^{2}w^{2}$.