Приведём пример работы алгоритма в кольце многочленов над конечным полем $\mathbb{Z}_{2}$, с мономиальным порядком $\mbox{degrevlex}(x>y>z>t)$.\\*
 \textbf{Инициализация:} IncrementalF5 записывает в массив $R$ входные многочлены, добавляя к ним сигнатуры с единичным мономом и индексом, равным порядковому номеру во входных данных:\\* $R\leftarrow[ [1]=\lpoly{\monone}{1}{x^{2}z+xyz},$\\*
$[2]=\lpoly{\monone}{2}{x^{2}y+y^{2}z}]$\\*
$G_2$ присваивается значение $\{2\}$.\\*
\textbf{Шаг 1:} AlgorithmF5 обрабатывает многочлен $R[1]$ и формирует по предыдущему базису $G_{1}=G_2\cup \{1\}$.\\*
CritPair: S-пара $(y\cdot R[1],z\cdot R[2])$ добавляется в $P$.\\*
AlgorithmF5: выбор S-пар степени 4.\\*
Spol: S-многочлен $y\cdot R[1] - z\cdot R[2]$ добавляется как $R[3]=\lpoly{y}{1}{xy^{2}z+y^{2}z^{2}}$ и попадает в правила, делая Rule[1] равным $[3]$.\\*
TopReduction: $R[3]$ не может быть далее редуцирован и добавляется в Done.\\*
AlgorithmF5: запись в $G_{1}$ позиций многочленов $\{3\}$.\\*
CritPair: S-пара $(x\cdot R[3],yz\cdot R[2])$ добавляется в $P$.\\*
CritPair: S-пара $(x\cdot R[3],y^{2}\cdot R[1])$ добавляется в $P$.\\*
AlgorithmF5: выбор S-пар степени 5.\\*
Spol: S-многочлен $x\cdot R[3] - yz\cdot R[2]$ добавляется как $R[4]=\lpoly{xy}{1}{xy^{2}z^{2}+y^{3}z^{2}}$ и попадает в правила, делая Rule[1] равным $[4, 3]$.\\*
Spol: S-многочлен $x\cdot R[3] - y^{2}\cdot R[1]$ отбрасывается критерием перезаписи, посольку сигнатура $x\cdot\Sss{y}{1}$ делится на сигнатуру элемента $R[4]$.\\*
TopReduction: $R[4]$ редуцирован редуктором $R[3]$ до $\lpoly{xy}{1}{y^{3}z^{2}+y^{2}z^{3}}$ и остаётся в ToDo.\\*
TopReduction: $R[4]$ не может быть далее редуцирован и добавляется в Done.\\*
AlgorithmF5: запись в $G_{1}$ позиций многочленов $\{4\}$.\\*
CritPair: S-пара $(x^{2}\cdot R[4],y^{2}z^{2}\cdot R[2])$  отбрасывается $\varphi$-проверкой, поскольку моном сигнатуры $x^{2}\cdot\Sss{xy}{1}$ делится на старший моном $x^{2}y$ из базиса прошлого шага.\\*
CritPair: S-пара $(x^{2}\cdot R[4],y^{3}z\cdot R[1])$  отбрасывается $\varphi$-проверкой, поскольку моном сигнатуры $x^{2}\cdot\Sss{xy}{1}$ делится на старший моном $x^{2}y$ из базиса прошлого шага.\\*
CritPair: S-пара $(x\cdot R[4],yz\cdot R[3])$  отбрасывается $\varphi$-проверкой, поскольку моном сигнатуры $x\cdot\Sss{xy}{1}$ делится на старший моном $x^{2}y$ из базиса прошлого шага.\\*
Результирующем базисом шага оказываются многочлены из R с позициями из $G_1=\{2, 1, 3, 4\}$.\\*
После убирания меток финальный нередуцированный базис принимает вид:\\*
$R[2]=x^{2}y+y^{2}z$\\*
$R[1]=x^{2}z+xyz$\\*
$R[3]=xy^{2}z+y^{2}z^{2}$\\*
$R[4]=y^{3}z^{2}+y^{2}z^{3}$