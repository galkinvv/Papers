Рассмотрим пример работы алгоритма в кольце многочленов над конечным полем $\mathbb{Z}_{2}$, с использованием мономиального порядка $\mbox{degrevlex}(x>y>z>t)$. \textbf{Инициализация:} IncrementalF5 записывает в массив $R$ входные многочлены, добавляя к ним сигнатуры с единичным мономом и индексом, равным порядковому номеру во входных данных:\\* $R\leftarrow[ [1]=\lpoly{\monone}{1}{xyz+t^{3}},$\\*$[2]=\lpoly{\monone}{2}{xy+xz+yz},$\\*$[3]=\lpoly{\monone}{3}{x+y+z}]$\\*$G\leftarrow\{3\}$\\*\textbf{Шаг 1:} AlgorithmF5 обрабатывает многочлен $\lpoly{\monone}{2}{xy+xz+yz}$.\\*\textbf{Шаг 2:} AlgorithmF5 обрабатывает многочлен $\lpoly{\monone}{1}{xyz+t^{3}}$.